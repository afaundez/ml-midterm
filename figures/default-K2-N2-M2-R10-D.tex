\pgfplotstableread[row sep=\\,col sep=&]{
values & d001 & d0001 & d00001 \\
1 & 0.6217987808286073 & 0.6137987267145945 & 0.6137987267145945 \\
2 & 0.6434589581321546 & 0.6137987267145945 & 0.6137987267145945 \\
3 & 0.6584925180705914 & 0.6137987267145943 & 0.6137987267145945 \\
4 & 0.6733906405321053 & 0.614054830134284 & 0.6137987267145945 \\
5 & 0.6879017987738398 & 0.6152165593472926 & 0.6137987267145945 \\
6 & 0.7008581900611026 & 0.6161483510408458 & 0.6137987267145943 \\
7 & 0.7106736380059988 & 0.6170681848943043 & 0.6137987267145946 \\
8 & 0.721965746261189 & 0.6181824597016432 & 0.6137987267145946 \\
9 & 0.731272428889093 & 0.6191439887513687 & 0.6137987267145945 \\
10 & 0.7389581277559807 & 0.6199583837680757 & 0.6137987267145945 \\
}\expectedgain

\begin{figure}
  \begin{tikzpicture}
    \begin{axis}[
        xtick={1,2,3,4,5,6,7,8,9,10},
        xlabel=Iteration Number,
        ylabel=Expected Gain,
        legend pos=north west
      ]
      \addplot table[x=values,y=d001]{\expectedgain};
      \addlegendentry{$\Delta = 10^{-2}$}
      \addplot table[x=values,y=d0001]{\expectedgain};
      \addlegendentry{$\Delta = 10^{-3}$}
      \addplot table[x=values,y=d00001]{\expectedgain};
      \addlegendentry{$\Delta = 10^{-4}$}
    \end{axis}
  \end{tikzpicture}
  \caption{Default case: Expected Gain increasing in each iteration using different sizes of $\Delta$ perturbations}
  \label{fig:results-K2-N2-M2-R10-D}
\end{figure}
