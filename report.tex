\documentclass[letterpaper, conference]{IEEEtran}

\usepackage{amsmath}
\usepackage{mathabx}
\DeclareMathOperator*{\argmax}{argmax}

\usepackage{pgfplots}
\pgfplotsset{compat=1.5.1}

%\usepackage{titlesec}
%\titlespacing*{\subsection}{0pt}{1.1\baselineskip}{\baselineskip}
%\setlength{\parindent}{2em}
%\setlength{\parskip}{0.7em}

\usepackage{url}

\begin{document}

\title{A Discrete Bayesian Classifier \\
  \large Machine Learning Midterm Report}

\author{
  \IEEEauthorblockN{Alvaro Faundez}
  \IEEEauthorblockA{
    \textit{Master in Data Science's Program, first year}\\
    \textit{CUNY Graduate Center}\\
    \textit{alvaro@faundez.net}
  }
}

\maketitle

\begin{abstract}

This report details the implementation of a discrete Bayesian classifier. The classifier requirements are a dataset and an economic gain matrix for input. The process calculates the class, measurement conditional, and class conditional probabilities from the dataset, all needed for the Bayes theorem. It outputs the Bayes rule, the confusion matrix, and the expected gain matrix associated with the input used. After completing the training and predicting steps, small perturbations are introduced in the measurement conditional probabilities of every misclassification, obtaining an improvement, as expected, in the expected gain (after recalculating all the classifier's outputs). The experiments included using the dataset split into equally sized training and test sets and using K-fold cross-validation. The dataset was created using synthetic probabilities distribution. A series of results are detailed in the appendix.

\end{abstract}

\section{Introduction}


The discrete Bayesian classifier is a popular predictive classifier broadly used in Machine Learning, based on the Bayes Theorem [\ref{eq:bayes-theorem}].

\begin{equation}\label{eq:bayes-theorem}
  P(A \mid B) = \frac{P(B \mid A)\mathbin{}P(A)}{P(B)}
\end{equation}

A discrete Bayesian classifier, given a set of classifications, a measurement space, and a dataset, uses the Bayes theorem to compute the conditional probabilities and the Bayes rule decision needed to perform classification. It can be optimized using an Economic Gain matrix, by maximizing the Expected Gain obtained using the Bayes decision rule.

In our case, given a set $C$ of $K$ discrete classifications [\ref{eq:classes-set}] and a discrete measurement space $D$, result of the cartesian product of $N$ discrete measurements $L_n$ [\ref{eq:measurement-space}], the classifier is a function assigning a unique classification $c \in C$ to any measurement $\vec{d} \in D$.

\begin{equation}\label{eq:classes-set}
C = \{\,c_1,\, \dots,\, c_{K}\,\}
\end{equation}


\begin{gather}\label{eq:measurement-space}
D = \bigtimes_{n=1}^{N} L_n \\
L_n = \{l_{n_1},\, \dots,\, l_{n_{M_n}}\},\, \forall n \in { 1, ..., N }, M_n = \vert L_n\vert
\end{gather}

For the classification, the following inputs are required:

\begin{itemize}
  \item the discrete classification cardinality, $K$
  \item the cardinalities of each discrete measurement, $\vert L_n \vert,\, n \in N$
  \item an Economic Gain Matrix, $e^{K \times K}$
  \item a dataset with $Z$ measurements with matching $Z$ classifications. 
\end{itemize}

The Economic Gain matrix, $e$, is a $K \times K$ matrix that defines the gain (or cost) of making the right (or wrong) classifications [\ref{eq:economic-gain-matrix}]. Each column will represent an assigned class, and each row will represent the true class. Then, every element $e(c_i,\, c_j) \in e$ represent the gain or cost of assigning the class $c_j$ when the true class was $c_i$. The economic gain usually is positive for every correct classification and non-positive for every other case. The identity matrix is an example of an economic gain matrix.

\begin{equation}\label{eq:economic-gain-matrix}
  e^{K \times K} = \begin{pmatrix}
    e(c_1,c_1) & \cdots & e(c_1,c_K) \\
      \vdots   & \ddots &    \vdots   \\
    e(c_K,c_1) & \cdots & e(c_K,c_K)
  \end{pmatrix}
\end{equation}


The economical consequences of the classifier are determined by the Expected Gain Matrix $G^{K \times K}$. Each element $g$ is the multiplication between the economic gain $e(i,\, j) \in K$ and the probability of the true class $c_i$ being assigned class $c_j$, $P(c_i,\, c_j)$ [\ref{eq:expected-gain-matrix}].

\begin{equation}\label{eq:expected-gain-matrix}
  G^{K \times K} = \begin{pmatrix}
    e(c_1,c_1)P(c_1,c_1) & \cdots & e(c_1,c_K)P(c_1,c_K) \\
                \vdots   & \ddots &    \vdots   \\
    e(c_K,c_1)P(c_K,c_1) & \cdots & e(c_K,c_K)P(c_K,c_K)
  \end{pmatrix}
\end{equation}

The Expected Gain of the classifier is the sum of all the elements of the Expected Gain Matrix [\ref{eq:expected-gain}].

\begin{equation}\label{eq:expected-gain}
  E[e] = \sum_{i \in K} \sum_{j \in K}e(c_i,\,c_k)\mathbin{}P(c_i,\,c_k)
\end{equation}

The goal is to determine the Bayes decision rule that maximizes the Expected Gain of the classifier, $E$.

A critical step in the construction of the classifier is the building of the Bayes decision rule $f_{\vec{d}}$ [\ref{eq:bayes-rule}]. The Bayes decision rule will assign 1 to the class that maximizes the Expected Gain [\ref{eq:bayes-argmax}], and 0 otherwise.

\begin{equation}\label{eq:bayes-rule}
  f_d(c_j) =
  \begin{cases}
  1 & j = k \\
  0 & j \neq k
  \end{cases}
\end{equation}

\begin{equation}\label{eq:bayes-argmax}
  \argmax_{c_k \in C} \sum_{j = 1}^{K} e(c_j, c_k)\mathbin{}P(c_j, \vec{d})
\end{equation}

The Expected Gain can be written in function of the Bayes decision rule [\label{eq:bayes-expected-gain}].

\begin{equation}\label{eq:bayes-expected-gain}
  E[e, f] = \sum_{i \in K} \sum_{j \in K}\sum_{\vec{d} \in D}f_{\vec{d}}(c_j)\mathbin{}e(c_i,\, c_j)\mathbin{}P(c_i,\,\vec{d})
\end{equation}

To obtain $f_{\vec{d}}$ and $P(c,\, \vec{d})$, it is necessary to calculate the posterior probability of assigning a class $c$ to measurement $\vec{d}$, $P(c \mid \vec{d})$, using the prior class probability $P(c)$ and the probability of the measurement given the class $P(\vec{d} \mid c)$ [\ref{eq:bayes-theorem-proportional}].

\begin{equation} \label{eq:bayes-theorem-proportional}
  P(c,\, \vec{d}) \mathbin{\propto} P(\vec{d} \mid c) \mathbin{} P(c)
\end{equation}

This report details an implementation of a discrete Bayesian classifier used to implement the classifier. A technical overview explains the design and implementation, along with the results of the experiments executed.

The definition and implementation include:

\begin{itemize}
  \item Classification and measurement dimensions, including pseudo-random probabilities and cumulative distribution functions
  \item Space definition and linear addresses
  \item Class prior and conditional probabilities
  \item Classifier optimization using an economic gain matrix
  \item Dataset definition and generation of pseudo-random synthetic data
  \item Experimentation framework with two types of cross-validation
\end{itemize}

The classifier's performance is tested by adding small perturbations to the class conditional, increasing the expected gain. The increase was tested using are training/test and k-folds cross-validation.

\section{Technical}

In order to explain the implementation, a few concepts must be discussed.

A experiment will be defined as the isolated instance that takes inputs and produce results. The experiments will defined dimensions, classifications, measurement space, classifier, and datasets.

A dimension is a 1-dimensional set of $M$ correlatives integer numbers from 1 to $K$. Any 1-dimensional $L$ set has a Probability Mass Function $P$ and a Cumulative Distribution Function $Q$. The $P$ set will be created by generating a set $R$ with $K$ pseudo-random numbers $r_1, ...r_K$ scaled to 1 $p_1, ...r_K$ [\ref{eq:pmf}]. The $Q$ set is sequence of cumulative sum of the probabilities in the $P$ [\ref{eq:cdf}].

\begin{equation}\label{eq:pmf}
  \begin{aligned}
  R &= \{r_1, \dots, r_K\} \\
  P &= \Bigl\{p_i \mid p_i = \frac{r_i}{r_{\sum_{i = 1}^{K} r_i}}, r_i \in R \Bigr\}
  \end{aligned}
\end{equation}

\begin{equation}\label{eq:cdf}
Q = \Bigl\{q_i \mid q_i = \sum_{j = 1}^{K} p_j, p_j \in P \Bigr\}
\end{equation}

The classifications set $C$ and every measurement $L_n$ are 1-dimensional set with their corresponding probabilities sets.

The cumulative distribution functions will be used to generate pseudo-random numbers in the 1-dimensional space: Given a cumulative distribution function $Q$ belonging to the dimension $M$, $f_Q$ will take a number $r \in [0, 1]$. $f_Q$ and assign it the number $k \in M$ according to the relative position of $r$ compared with the numbers in $Q$ [\ref{eq:cummulative}].

\begin{equation}\label{eq:cummulative}
  f_Q(r) =
  \begin{cases}
  1 & r < q_1 \\
  k & r q_{k - 1} \leq q_k,\, k \in M - \{1\} \\
  \end{cases}
\end{equation}

The measurement space $D$ is the cartesian product of a set of measurements \label{eq:measurement-space}. Every element in $\vec{d} \in D$ is a vector where every value corresponds to a measurement value.

\begin{equation}
  \vec{d} = (d_0, ..., d_N), \vec{d}_i \in L_n
\end{equation}

A sample dataset $XY$ is formed by two sets of equal size $Z$: the data $ X = \{x_i \in D \mid i \in [1, Z]\}$ and the target $Y = \{ y_i \in C \mid i \in [1, Z]\}$.






\subsection{Requirements}

To run the code, a Unix environment with Ruby 2.6+ is necessary. The code is available on GitHub at \url{https://github.com/afaundez/ml-midterm}.

To test the command and check the available options run:

\begin{verbatim}
  $ ./midterm --help
\end{verbatim}

\subsection{options}

For testing, the following parameters are available for configuration:

\begin{itemize}
    \item-s, --seed [INT]                 Pseudo-random seed, an integer. Default: nil
    \item-c, --classes [INT]              Class cardinality, an integer. Default 2
    \item-m, --measurements [INT]         Measurements size, an integer. Default 5
    \item    --measurement-min-cardinality [INT]
                                     Measurement Min Cardinality, an integer. Default 3
    \item    --measurement-max-cardinality [INT]
                                     Measurement Max Cardinality, an integer. Default 6
    \item    --sample-size [INT]          Sample size, an integer. Default to 10 times the  space addresses size.
    \item-i, --iterations [INT]           Iterations, an integer. Default 2
    \item-d, --delta [FLOAT]              Delta for conditionals improvement, a float. Default: 0.01
    \item    --no-overlap                 Generate classes based on measurements
    \item    --uniform                    Use uniform distribution on all dimensions
\end{itemize}

\subsection{Runtime}

\subsubsection{Build}

During the build, three significant abstractions are used: Dimension, Space, and DataSet.

The Dimension abstraction is meant to store the cardinality, probabilities distribution function, and cumulative distribution function of a single measurement or class. A dimension does not store values, but it can be used to generate random values using the distribution functions. A Dimension is defined by:

\begin{itemize}
  \item size
  \item pdf, generated based on the size
  \item cdf, generated based on the pdf
  \item distribution, whether to use a random or uniform distribution
\end{itemize}

The Space abstraction stores the single class dimension \ref{classes}, the collection of measurements dimensions \ref{dimensions}, and the class conditional probabilities. Since a space knows all the dimensions specifications, it is in charge of translating and measurement into a linear address and vice-versa. A space is defined by:

\begin{itemize}
\item class\_dimension, a Dimension instance
\item measurements\_dimensions, a collection of Dimension instances
\item likelihoods, random pdfs
\end{itemize}

The DataSet abstraction generates and stores measurement values and the associated classification, all bounded to a specific space. Since it has access to the data and space, it is in charge of the training and improving the class conditional probabilities, determining the Bayes Decision Rule, generating the Confusion Matrix, and calculating the Expected Gain. It is relevant that at the time of creating, the prior class probabilities are calculated and stored in the DataSet. A DataSet is defined by:

\begin{itemize}
\item size
\item space, a Space instance
\item samples, a collection of collections of measurements
\item class\_outcomes, counter of classes for prior class probabilities calculation
\item overlap, whether to assign classes with overlap
\end{itemize}

The Economic Gain Matrix is generated in this step. Even though it is a method, currently, the only matrix available is the identity matrix.

\subsubsection{Training}

Once a set of DataSet is created, the test DataSet is picked and left isolated, and the rest is used as train DataSet.

The training process consists of two steps. Using the training set and the economic gain matrix, the training outputs the Bayes decision rule and the expected gain. Then, using the Bayes decision rule, it generates the confusion matrix.


At the end of this process, the Space associated with the DataSet will store the measurement conditional probabilities $P(c \mid d)$ targeted in \ref{eq:bayes}.

\subsection{Optimization}

Now that the space is trained and the measurement conditional probabilities have been processed, using the delta as input, the dataset can improve the class conditional probabilities using the test DataSet. For each mismatched classification, the corresponding class conditional probability is incremented by the delta input, and the measurement probabilities are normalized to 1. This process yields an updated Space and accuracy obtained.

\section{Experimental Results}

Even though it is not a strict test, it is good to do a sanity check with a dummy case: one feature, one value measurement. It is expected to have an expected gain of 1.0 and an accuracy of 1.0.

Running

\begin{verbatim}
  bin/midterm --classes 1 --measurements 1
\end{verbatim}

yields, as expected,

\begin{verbatim}
Economic Gain Matrix
+---+
| 1 |
+---+

Train 0.
  Expected Gain: 1.0
  Confusion Matrix Trace: 1.0
  Accuracy: 1.0
Confusion Matrix
+-----+
| 1.0 |
+-----+

Test
  Expected Gain: 1.0
  Confusion Matrix Trace: 1.0
  Accuracy: 1.0
\end{verbatim}


The next step is to check the results using. Let us use two classes, five values measurements, and ten training sets, and start setting the seed number in order to have replicable experiments.

\subsection{First results}

\begin{verbatim}
  bin/midterm --classes 2 --measurements 5 \
    --iterations 10 --seed 1234
\end{verbatim}

The expected gain and accuracy do not seem to be correct

\begin{figure}[hbt]
  \label{fig:10-training-2-classes}
  \caption{}
\end{figure}

Incrementing the training iterations to 100 does not make real improvements; it keeps going up and down.

\begin{figure}[hbt]
  \label{fig:100-training-2-classes}
  \caption{}
\end{figure}

To identify the reason of these results, two options were added to the application, one to control the overlap in the class assignment to a measurement and how the probability distribution is generated.

\subsection{Probability distributions}

Since the method used to generate the probability distribution function for a dimension does not generate even probabilities for each value in the dimension, a parameter $--uniform$ was added to revert this and generate uniform distribution for the classes and measurement. The hypothesis was that by generating numbers evenly, the results would easier to understand.

Now, repeating the previous commands with the new parameter:

\begin{verbatim}
  bin/midterm --classes 2 --measurements 5 \
    --iterations 10 --seed 1234 \
    --uniform
\end{verbatim}

\begin{figure}[hbt]
  \label{fig:10-training-2-classes-uniform}
  \caption{}
\end{figure}

The expected gain does not seem to be improving, but the accuracy is set around $0.5$. It could be a sign that accuracy is bouncing between that particular number because now each class has the same probability of being assign to a measurement. Using 100 training iterations, the result is similar.

\begin{figure}[hbt]
  \label{fig:100-training-2-classes-uniform}
  \caption{}
\end{figure}

Checking with 3 classes, the accuracy bounce around $0.33$ as expected:

\begin{figure}[hbt]
  \label{fig:10-training-3-classes-uniform}
  \caption{}
\end{figure}

\subsection{Class overlapping}

To understand if the accuracies shown in the previous steps are determined by the number of classes and the overlapping caused by an assignation independent from the measurements, it was added to the program the option to avoid the overlap and assign a class based on the norm of the measurement vector. This is an arbitrary assignment chosen just by the simplicity and certainty of the assignment's uniqueness.

using the new parameter

\begin{verbatim}
  bin/midterm --classes 2 --measurements 5 \
    --iterations 10 --seed 1234 \
    --no-overlap
\end{verbatim}

\begin{figure}[hbt]
  \label{fig:10-training-2-classes-no-overlapping}
  \caption{}
\end{figure}

Now the expected gain is considerably higher, and trying with 100 iterations stays close to 1.0.

\begin{figure}[hbt]
  \label{fig:100-training-2-classes-no-overlapping}
  \caption{}
\end{figure}

\section{Conclusions}

This report shows a Bayes classifier written from scratch in Ruby without any scientific library, following the theory and instructions provided during the Machine Learning class.

The program includes definitions and abstractions for Dimension, Space, and DataSet that can be configured with optional parameters through the command line to create classes, measurement, and training sets.

The initial results were not the one expected, showing low and bouncing expected gains and accuracy. As a way to explain these behaviors, two paths were developed

\begin{itemize}
  \item First, create the dimensions for classes and measurements using uniform probability distribution functions, expecting more recognizable numbers. The method worked, and it allowed us to determine that the accuracy was inversely proportional to the classes' cardinality.
  \item Then, with the conclusion from the previous step, it was introduced an option to assign a class to a specific measurement without overlapping. The norm of the measurement vector was the method used, and it produced a considerable impact on the expected gain and accuracy, reaching values over $0.99$ in both parameters.
\end{itemize}

Although some aspects of the value could have been explained with the overlapping of classes, it remains inconclusive why the expected gain and accuracy are not reaching 1.0 or why it does not stop bouncing.

figures/test-c10-m4-l4-r10
\pgfplotstableread[row sep=\\,col sep=&]{
values & P \\
0 & 0.0712096066704804 \\
1 & 0.095335789832152 \\
2 & 0.0239558678029414 \\
3 & 0.03380437761212291 \\
4 & 0.13664042554857556 \\
5 & 0.16522581472824513 \\
6 & 0.053483238920656165 \\
7 & 0.1181391176694822 \\
8 & 0.14954854694045272 \\
9 & 0.15265721427489157 \\
}\priors

\begin{tikzpicture}
  \begin{axis}[ybar]
    \addplot table[x=values,y=P]{\priors};
  \end{axis}
\end{tikzpicture}
\pgfplotstableread[row sep=\\,col sep=&]{
values & P \\
0 & 0.7318810568515274 \\
1 & 0.8064409137923557 \\
2 & 0.8318360754136418 \\
3 & 0.8497881829488712 \\
4 & 0.8603903047663723 \\
5 & 0.8672001571481214 \\
6 & 0.8725972992541466 \\
7 & 0.8773463938509242 \\
8 & 0.8808405793653348 \\
9 & 0.88600897735109 \\
}\expectedgaincenti

\pgfplotstableread[row sep=\\,col sep=&]{
values & P \\
0 & 0.7804722456765985 \\
1 & 0.9041750170048655 \\
2 & 0.9341089042261965 \\
3 & 0.9449644220024328 \\
4 & 0.9583408607694168 \\
5 & 0.9653423113024769 \\
6 & 0.9679846067772699 \\
7 & 0.9726859119551611 \\
8 & 0.9751773330665396 \\
9 & 0.9767596589962781 \\
}\expectedgaindeci

\pgfplotstableread[row sep=\\,col sep=&]{
values & P \\
0 & 0.5525353589016282 \\
1 & 0.6303661708231101 \\
2 & 0.6733976302127822 \\
3 & 0.700551265445668 \\
4 & 0.7193527473728801 \\
5 & 0.7333401984043427 \\
6 & 0.7450626950111878 \\
7 & 0.7542905601793841 \\
8 & 0.761355571412957 \\
9 & 0.7678184935144412 \\
}\expectedgainmili

\pgfplotstableread[row sep=\\,col sep=&]{
values & P \\
0 & 0.321779870985538 \\
1 & 0.3682579874074543 \\
2 & 0.40476008070689357 \\
3 & 0.4347112186811146 \\
4 & 0.4595766056499592 \\
5 & 0.48121953269770834 \\
6 & 0.500370667432365 \\
7 & 0.5169828618890291 \\
8 & 0.5317776052006948 \\
9 & 0.5449640599692005 \\
}\expectedgaindecimili

\begin{tikzpicture}
  \begin{axis}
    \addplot table[x=values,y=P]{\expectedgaindeci};
    \addplot table[x=values,y=P]{\expectedgaincenti};
    \addplot table[x=values,y=P]{\expectedgainmili};
    \addplot table[x=values,y=P]{\expectedgaindecimili};
  \end{axis}
\end{tikzpicture}


figures/test-c10-m4-l4-r100
\pgfplotstableread[row sep=\\,col sep=&]{
values & P \\
0 & 0.0712096066704804 \\
1 & 0.095335789832152 \\
2 & 0.0239558678029414 \\
3 & 0.03380437761212291 \\
4 & 0.13664042554857556 \\
5 & 0.16522581472824513 \\
6 & 0.053483238920656165 \\
7 & 0.1181391176694822 \\
8 & 0.14954854694045272 \\
9 & 0.15265721427489157 \\
}\priors

\begin{tikzpicture}
  \begin{axis}[ybar]
    \addplot table[x=values,y=P]{\priors};
  \end{axis}
\end{tikzpicture}

\pgfplotstableread[row sep=\\,col sep=&]{
values & P \\
0 & 0.7804722456765985 \\
1 & 0.9041750170048655 \\
2 & 0.9341089042261965 \\
3 & 0.9449644220024328 \\
4 & 0.9583408607694168 \\
5 & 0.9653423113024769 \\
6 & 0.9679846067772699 \\
7 & 0.9726859119551611 \\
8 & 0.9751773330665396 \\
9 & 0.9767596589962781 \\
10 & 0.9780498102550484 \\
11 & 0.9785990696161182 \\
12 & 0.9799936294720202 \\
13 & 0.9800048997149003 \\
14 & 0.9866956932206437 \\
15 & 0.9876716480101324 \\
16 & 0.9879898561470783 \\
17 & 0.9885321272826154 \\
18 & 0.9885340933824205 \\
19 & 0.9890888999948637 \\
20 & 0.9890903892579024 \\
21 & 0.9893800342925713 \\
22 & 0.9898739949812988 \\
23 & 0.9905542276729212 \\
24 & 0.9909889554540445 \\
25 & 0.991434672359282 \\
26 & 0.9916165420984256 \\
27 & 0.99162132817829 \\
28 & 0.9927043892959463 \\
29 & 0.9927049820838225 \\
30 & 0.9928568069222904 \\
31 & 0.992857994708866 \\
32 & 0.992858264434907 \\
33 & 0.9931175327630337 \\
34 & 0.9931177369508051 \\
35 & 0.9931177369508655 \\
36 & 0.9931179225753788 \\
37 & 0.9934316821195233 \\
38 & 0.9934316821195237 \\
39 & 0.9934318355282099 \\
40 & 0.9935485665175322 \\
41 & 0.9938579549740897 \\
42 & 0.9939663708705696 \\
43 & 0.9939664589150407 \\
44 & 0.9939666056558254 \\
45 & 0.9939666723561821 \\
46 & 0.9940629329414539 \\
47 & 0.994063044108715 \\
48 & 0.994249315497872 \\
49 & 0.9944168234010624 \\
50 & 0.9945152577485848 \\
51 & 0.9945152577485848 \\
52 & 0.994515341966207 \\
53 & 0.9946847178734506 \\
54 & 0.9949911845583865 \\
55 & 0.9949911845583865 \\
56 & 0.9952189055396474 \\
57 & 0.9953064419323681 \\
58 & 0.9954466117259541 \\
59 & 0.995526165928658 \\
60 & 0.995526210544902 \\
61 & 0.9956073184227122 \\
62 & 0.9956073184227122 \\
63 & 0.9957329304773981 \\
64 & 0.9957329472378204 \\
65 & 0.9957329472378204 \\
66 & 0.9957329472378204 \\
67 & 0.9959748231249246 \\
68 & 0.9961903855594441 \\
69 & 0.9962621630067252 \\
70 & 0.996335597645694 \\
71 & 0.9963355976456942 \\
72 & 0.9963356058906873 \\
73 & 0.9963356058906873 \\
74 & 0.9963356058906873 \\
75 & 0.9963356058906873 \\
76 & 0.9963356133861356 \\
77 & 0.9964382355134258 \\
78 & 0.9964382355134258 \\
79 & 0.9964382417080112 \\
80 & 0.9964382473394525 \\
81 & 0.9965050061021514 \\
82 & 0.9965050190977852 \\
83 & 0.9966175309584737 \\
84 & 0.9966175381782703 \\
85 & 0.9966175414599959 \\
86 & 0.9966175414599959 \\
87 & 0.9967156834592443 \\
88 & 0.9967156834592441 \\
89 & 0.9968054371791184 \\
90 & 0.9968054371791184 \\
91 & 0.9968054371791184 \\
92 & 0.9968054371791184 \\
93 & 0.9968983259760184 \\
94 & 0.9968983259760185 \\
95 & 0.9968983259760184 \\
96 & 0.9968983286881884 \\
97 & 0.9968983311537974 \\
98 & 0.9968983311537974 \\
99 & 0.996963009734322 \\
}\expectedgaindeci

\pgfplotstableread[row sep=\\,col sep=&]{
values & P \\
0 & 0.7318810568515274 \\
1 & 0.8064409137923557 \\
2 & 0.8318360754136418 \\
3 & 0.8497881829488712 \\
4 & 0.8603903047663723 \\
5 & 0.8672001571481214 \\
6 & 0.8725972992541466 \\
7 & 0.8773463938509242 \\
8 & 0.8808405793653348 \\
9 & 0.88600897735109 \\
10 & 0.8890039067192418 \\
11 & 0.8920505858546528 \\
12 & 0.894812559193665 \\
13 & 0.8964379605958694 \\
14 & 0.8979091386004169 \\
15 & 0.9001831566365857 \\
16 & 0.9014211544161178 \\
17 & 0.9028202824799675 \\
18 & 0.9050786045206501 \\
19 & 0.9062927636355005 \\
20 & 0.907490633939971 \\
21 & 0.9098004412660313 \\
22 & 0.9108421178460887 \\
23 & 0.9120617239970098 \\
24 & 0.9131677028037167 \\
25 & 0.9149735365997546 \\
26 & 0.916131938184022 \\
27 & 0.9168175911717373 \\
28 & 0.9176505055048291 \\
29 & 0.9181085783840678 \\
30 & 0.9191631646712053 \\
31 & 0.9199436409872732 \\
32 & 0.920535081098255 \\
33 & 0.9211401172123324 \\
34 & 0.9219730515052169 \\
35 & 0.9222961269288554 \\
36 & 0.9227191221926258 \\
37 & 0.9235469778174484 \\
38 & 0.9244792496129794 \\
39 & 0.9247559110692608 \\
40 & 0.9250504338349075 \\
41 & 0.9258131003611296 \\
42 & 0.9263435072757619 \\
43 & 0.9265942001221645 \\
44 & 0.9274603546671744 \\
45 & 0.9278080444088647 \\
46 & 0.9282657401434875 \\
47 & 0.9286273643109493 \\
48 & 0.9289127315682828 \\
49 & 0.9296027611629202 \\
50 & 0.9310535932511378 \\
51 & 0.9317553386605736 \\
52 & 0.9321681133472998 \\
53 & 0.9328317274566195 \\
54 & 0.9330266617406588 \\
55 & 0.9333704806808549 \\
56 & 0.9336624584745666 \\
57 & 0.9337354461987574 \\
58 & 0.934125453074106 \\
59 & 0.9377300943913258 \\
60 & 0.9383313350846472 \\
61 & 0.9384989361275504 \\
62 & 0.9388510143338802 \\
63 & 0.9396269525560331 \\
64 & 0.9399411496151912 \\
65 & 0.9399959793979537 \\
66 & 0.9402987805397384 \\
67 & 0.9407866996122884 \\
68 & 0.9416958510418566 \\
69 & 0.9421002519483714 \\
70 & 0.942737842423203 \\
71 & 0.9430108842558264 \\
72 & 0.943547975885601 \\
73 & 0.9438564527560496 \\
74 & 0.9439984901702071 \\
75 & 0.944759165143782 \\
76 & 0.9450164360193217 \\
77 & 0.9451943956596859 \\
78 & 0.9455121769070136 \\
79 & 0.9461434219240871 \\
80 & 0.9469071811407265 \\
81 & 0.9473936928152475 \\
82 & 0.947564123399667 \\
83 & 0.9480275901344433 \\
84 & 0.9482095988695031 \\
85 & 0.9483149603088457 \\
86 & 0.9485445424273959 \\
87 & 0.9486198157432828 \\
88 & 0.9486726964547872 \\
89 & 0.9486978216549063 \\
90 & 0.9489173062753641 \\
91 & 0.9493561439188403 \\
92 & 0.949544663741213 \\
93 & 0.9496269681541051 \\
94 & 0.9496280837452212 \\
95 & 0.9496486164281582 \\
96 & 0.9497254987311714 \\
97 & 0.9498682515926005 \\
98 & 0.9502007501015447 \\
99 & 0.950268762558764 \\
}\expectedgaincenti

\pgfplotstableread[row sep=\\,col sep=&]{
values & P \\
0 & 0.321779870985538 \\
1 & 0.3682579874074543 \\
2 & 0.40476008070689357 \\
3 & 0.4347112186811146 \\
4 & 0.4595766056499592 \\
5 & 0.48121953269770834 \\
6 & 0.500370667432365 \\
7 & 0.5169828618890291 \\
8 & 0.5317776052006948 \\
9 & 0.5449640599692005 \\
10 & 0.5567459166062645 \\
11 & 0.5674441612270988 \\
12 & 0.5770997945808702 \\
13 & 0.585869652534473 \\
14 & 0.5937529889452866 \\
15 & 0.6010748289809272 \\
16 & 0.6081089777007506 \\
17 & 0.6142768347192422 \\
18 & 0.6200961151229315 \\
19 & 0.6256081559300106 \\
20 & 0.6308337288484878 \\
21 & 0.6355483748873685 \\
22 & 0.6400173260676874 \\
23 & 0.6442340199361113 \\
24 & 0.6481891351502951 \\
25 & 0.6518971376323834 \\
26 & 0.6555337978376142 \\
27 & 0.6589307790106955 \\
28 & 0.6621626540831138 \\
29 & 0.6654760374830611 \\
30 & 0.6685101393577076 \\
31 & 0.6712082824336638 \\
32 & 0.6737886101662547 \\
33 & 0.6763577530918088 \\
34 & 0.6788938914288662 \\
35 & 0.6813162138491595 \\
36 & 0.6835885345970074 \\
37 & 0.6857486744355538 \\
38 & 0.6877737803424927 \\
39 & 0.6897715810885605 \\
40 & 0.6916709183293519 \\
41 & 0.6935206301298041 \\
42 & 0.695280601330618 \\
43 & 0.6969757889594983 \\
44 & 0.6986374375879498 \\
45 & 0.7001828379701659 \\
46 & 0.7017193462948019 \\
47 & 0.703252790064599 \\
48 & 0.7046750760049965 \\
49 & 0.7060167698081424 \\
50 & 0.7074672314595969 \\
51 & 0.7087623270620357 \\
52 & 0.7100332806817122 \\
53 & 0.7113146025317255 \\
54 & 0.7125215744559413 \\
55 & 0.7136440061502074 \\
56 & 0.7148369602175586 \\
57 & 0.7159882312816266 \\
58 & 0.7170008978599165 \\
59 & 0.71808242688287 \\
60 & 0.7190612867489139 \\
61 & 0.7200142337671478 \\
62 & 0.7209555340599951 \\
63 & 0.7218396069461779 \\
64 & 0.722791394276443 \\
65 & 0.7237611140292465 \\
66 & 0.7246775347468359 \\
67 & 0.7255221377071334 \\
68 & 0.7264090085623026 \\
69 & 0.7272318243473332 \\
70 & 0.7280763945459429 \\
71 & 0.7288101636722905 \\
72 & 0.7295710868269162 \\
73 & 0.7303464836180832 \\
74 & 0.7310893650417958 \\
75 & 0.7317877368593178 \\
76 & 0.7325069693865889 \\
77 & 0.7332085836589892 \\
78 & 0.7338908270286247 \\
79 & 0.7345513155736214 \\
80 & 0.7352015852513154 \\
81 & 0.7358515105193848 \\
82 & 0.7364853521047823 \\
83 & 0.7370993024339578 \\
84 & 0.7377083439441072 \\
85 & 0.7382815282775341 \\
86 & 0.7388013530092726 \\
87 & 0.7393924716163941 \\
88 & 0.7399369200295377 \\
89 & 0.7404595338546547 \\
90 & 0.7410240433442297 \\
91 & 0.7415975676521259 \\
92 & 0.7421382308824609 \\
93 & 0.7426830552371498 \\
94 & 0.7431879933445441 \\
95 & 0.7436731014086198 \\
96 & 0.7441520934020505 \\
97 & 0.7446557019131931 \\
98 & 0.7450823054354648 \\
99 & 0.7455276863801695 \\
}\expectedgaindecimili

\pgfplotstableread[row sep=\\,col sep=&]{
values & P \\
0 & 0.5525353589016282 \\
1 & 0.6303661708231101 \\
2 & 0.6733976302127822 \\
3 & 0.700551265445668 \\
4 & 0.7193527473728801 \\
5 & 0.7333401984043427 \\
6 & 0.7450626950111878 \\
7 & 0.7542905601793841 \\
8 & 0.761355571412957 \\
9 & 0.7678184935144412 \\
10 & 0.7728177320316557 \\
11 & 0.7773556125943928 \\
12 & 0.781404779293789 \\
13 & 0.7847268669783732 \\
14 & 0.7875649205089057 \\
15 & 0.7902448286385246 \\
16 & 0.7928970153601597 \\
17 & 0.7951930209001534 \\
18 & 0.7971987368691178 \\
19 & 0.7994997957705303 \\
20 & 0.8017001815301597 \\
21 & 0.8034331848002574 \\
22 & 0.8050980667816945 \\
23 & 0.8070905419892743 \\
24 & 0.8086462264162598 \\
25 & 0.8096349128749908 \\
26 & 0.8106866425543535 \\
27 & 0.8118488329987493 \\
28 & 0.8127420208421681 \\
29 & 0.8138696502833029 \\
30 & 0.815282859311553 \\
31 & 0.8160711693715088 \\
32 & 0.8172649168543369 \\
33 & 0.8182155657275396 \\
34 & 0.8193674383747225 \\
35 & 0.8203664562138607 \\
36 & 0.8211503822396339 \\
37 & 0.8217483905261278 \\
38 & 0.8225791251337828 \\
39 & 0.8234013741321017 \\
40 & 0.8241641213403221 \\
41 & 0.8251446415770314 \\
42 & 0.8258935805417025 \\
43 & 0.8265779121424245 \\
44 & 0.8272543663705381 \\
45 & 0.8278432336053024 \\
46 & 0.8283658905188441 \\
47 & 0.8292908221993263 \\
48 & 0.8300462888773519 \\
49 & 0.8306779540475859 \\
50 & 0.8314364154290099 \\
51 & 0.83184711032664 \\
52 & 0.832513876577025 \\
53 & 0.8331707869327346 \\
54 & 0.8337397518348467 \\
55 & 0.8343608420321497 \\
56 & 0.8345729680561367 \\
57 & 0.8351029335043818 \\
58 & 0.8354869045956551 \\
59 & 0.8360225037460033 \\
60 & 0.8366079450307389 \\
61 & 0.8374397847670015 \\
62 & 0.838183315027079 \\
63 & 0.8389193682815654 \\
64 & 0.8392872018483224 \\
65 & 0.839845593386068 \\
66 & 0.8402693068079006 \\
67 & 0.8407297752426408 \\
68 & 0.8413971966370899 \\
69 & 0.8417342547104733 \\
70 & 0.842169822820502 \\
71 & 0.8424633988835081 \\
72 & 0.842986181112338 \\
73 & 0.8435378572254534 \\
74 & 0.8438263179119309 \\
75 & 0.8440560778066112 \\
76 & 0.8444478640931247 \\
77 & 0.8448543111030067 \\
78 & 0.8451948947434282 \\
79 & 0.8456666185919275 \\
80 & 0.8461843897767908 \\
81 & 0.8465647385888184 \\
82 & 0.8470135696413481 \\
83 & 0.8474408926100111 \\
84 & 0.847878065762134 \\
85 & 0.8482506855199055 \\
86 & 0.848686102106634 \\
87 & 0.8491466716329366 \\
88 & 0.8495107752294625 \\
89 & 0.8497147085917265 \\
90 & 0.850075175274582 \\
91 & 0.8506924801888621 \\
92 & 0.8512662226320415 \\
93 & 0.8517776630213825 \\
94 & 0.8521269205418689 \\
95 & 0.8522665305159297 \\
96 & 0.8524729165509792 \\
97 & 0.8530307686850053 \\
98 & 0.8533947103783032 \\
99 & 0.8535788635513728 \\
}\expectedgainmili

\begin{tikzpicture}
  \begin{axis}
    \addplot table[x=values,y=P]{\expectedgaindeci};
    \addplot table[x=values,y=P]{\expectedgaincenti};
    \addplot table[x=values,y=P]{\expectedgainmili};
    \addplot table[x=values,y=P]{\expectedgaindecimili};
  \end{axis}
\end{tikzpicture}


figures/test-c10-m4-l4-r1000
\pgfplotstableread[row sep=\\,col sep=&]{
values & P \\
0 & 0.0712096066704804 \\
1 & 0.095335789832152 \\
2 & 0.0239558678029414 \\
3 & 0.03380437761212291 \\
4 & 0.13664042554857556 \\
5 & 0.16522581472824513 \\
6 & 0.053483238920656165 \\
7 & 0.1181391176694822 \\
8 & 0.14954854694045272 \\
9 & 0.15265721427489157 \\
}\priors

\begin{tikzpicture}
  \begin{axis}[ybar]
    \addplot table[x=values,y=P]{\priors};
  \end{axis}
\end{tikzpicture}
\pgfplotstableread[row sep=\\,col sep=&]{
values & P \\
0 & 0.7804722456765985 \\
1 & 0.9041750170048655 \\
2 & 0.9341089042261965 \\
3 & 0.9449644220024328 \\
4 & 0.9583408607694168 \\
5 & 0.9653423113024769 \\
6 & 0.9679846067772699 \\
7 & 0.9726859119551611 \\
8 & 0.9751773330665396 \\
9 & 0.9767596589962781 \\
10 & 0.9780498102550484 \\
11 & 0.9785990696161182 \\
12 & 0.9799936294720202 \\
13 & 0.9800048997149003 \\
14 & 0.9866956932206437 \\
15 & 0.9876716480101324 \\
16 & 0.9879898561470783 \\
17 & 0.9885321272826154 \\
18 & 0.9885340933824205 \\
19 & 0.9890888999948637 \\
20 & 0.9890903892579024 \\
21 & 0.9893800342925713 \\
22 & 0.9898739949812988 \\
23 & 0.9905542276729212 \\
24 & 0.9909889554540445 \\
25 & 0.991434672359282 \\
26 & 0.9916165420984256 \\
27 & 0.99162132817829 \\
28 & 0.9927043892959463 \\
29 & 0.9927049820838225 \\
30 & 0.9928568069222904 \\
31 & 0.992857994708866 \\
32 & 0.992858264434907 \\
33 & 0.9931175327630337 \\
34 & 0.9931177369508051 \\
35 & 0.9931177369508655 \\
36 & 0.9931179225753788 \\
37 & 0.9934316821195233 \\
38 & 0.9934316821195237 \\
39 & 0.9934318355282099 \\
40 & 0.9935485665175322 \\
41 & 0.9938579549740897 \\
42 & 0.9939663708705696 \\
43 & 0.9939664589150407 \\
44 & 0.9939666056558254 \\
45 & 0.9939666723561821 \\
46 & 0.9940629329414539 \\
47 & 0.994063044108715 \\
48 & 0.994249315497872 \\
49 & 0.9944168234010624 \\
50 & 0.9945152577485848 \\
51 & 0.9945152577485848 \\
52 & 0.994515341966207 \\
53 & 0.9946847178734506 \\
54 & 0.9949911845583865 \\
55 & 0.9949911845583865 \\
56 & 0.9952189055396474 \\
57 & 0.9953064419323681 \\
58 & 0.9954466117259541 \\
59 & 0.995526165928658 \\
60 & 0.995526210544902 \\
61 & 0.9956073184227122 \\
62 & 0.9956073184227122 \\
63 & 0.9957329304773981 \\
64 & 0.9957329472378204 \\
65 & 0.9957329472378204 \\
66 & 0.9957329472378204 \\
67 & 0.9959748231249246 \\
68 & 0.9961903855594441 \\
69 & 0.9962621630067252 \\
70 & 0.996335597645694 \\
71 & 0.9963355976456942 \\
72 & 0.9963356058906873 \\
73 & 0.9963356058906873 \\
74 & 0.9963356058906873 \\
75 & 0.9963356058906873 \\
76 & 0.9963356133861356 \\
77 & 0.9964382355134258 \\
78 & 0.9964382355134258 \\
79 & 0.9964382417080112 \\
80 & 0.9964382473394525 \\
81 & 0.9965050061021514 \\
82 & 0.9965050190977852 \\
83 & 0.9966175309584737 \\
84 & 0.9966175381782703 \\
85 & 0.9966175414599959 \\
86 & 0.9966175414599959 \\
87 & 0.9967156834592443 \\
88 & 0.9967156834592441 \\
89 & 0.9968054371791184 \\
90 & 0.9968054371791184 \\
91 & 0.9968054371791184 \\
92 & 0.9968054371791184 \\
93 & 0.9968983259760184 \\
94 & 0.9968983259760185 \\
95 & 0.9968983259760184 \\
96 & 0.9968983286881884 \\
97 & 0.9968983311537974 \\
98 & 0.9968983311537974 \\
99 & 0.996963009734322 \\
100 & 0.9969630119757846 \\
101 & 0.9969630119757846 \\
102 & 0.9969630119757847 \\
103 & 0.9969630140134781 \\
104 & 0.9970142662874358 \\
105 & 0.9970990740154851 \\
106 & 0.9970990740154851 \\
107 & 0.9970990740154851 \\
108 & 0.9970990755464344 \\
109 & 0.9970990755464344 \\
110 & 0.9970990755464344 \\
111 & 0.9970990755464344 \\
112 & 0.9970990755464344 \\
113 & 0.9970990755464344 \\
114 & 0.9971803095458144 \\
115 & 0.997239108255382 \\
116 & 0.9972391082553821 \\
117 & 0.9972391095206293 \\
118 & 0.9972391095206293 \\
119 & 0.9973134634564627 \\
120 & 0.9973134634564625 \\
121 & 0.9973134634564625 \\
122 & 0.9973814549223023 \\
123 & 0.9973814549223023 \\
124 & 0.9974578264157099 \\
125 & 0.99745782641571 \\
126 & 0.9974578272870924 \\
127 & 0.9975042438740644 \\
128 & 0.99754664727885 \\
129 & 0.9975853414134592 \\
130 & 0.997638587618117 \\
131 & 0.9976385882182426 \\
132 & 0.9977083774917603 \\
133 & 0.9977083774917603 \\
134 & 0.9977972715378672 \\
135 & 0.9977972720338387 \\
136 & 0.9977972720338387 \\
137 & 0.997797272484722 \\
138 & 0.997860988520789 \\
139 & 0.9978609895612889 \\
140 & 0.9978609898765919 \\
141 & 0.997860990163231 \\
142 & 0.9978961098301711 \\
143 & 0.9979544862490158 \\
144 & 0.9979544862490158 \\
145 & 0.9979544862490158 \\
146 & 0.9979544862490158 \\
147 & 0.9979544865095967 \\
148 & 0.9979865514181685 \\
149 & 0.9979865514181685 \\
150 & 0.9979865514181685 \\
151 & 0.9979865514181685 \\
152 & 0.9979865514181685 \\
153 & 0.9980158771117454 \\
154 & 0.9980158771117454 \\
155 & 0.9981295335893462 \\
156 & 0.9981295338047024 \\
157 & 0.9981295338047024 \\
158 & 0.9981295338047024 \\
159 & 0.9981295340004808 \\
160 & 0.9981295340004807 \\
161 & 0.9981561938998946 \\
162 & 0.9981561940616949 \\
163 & 0.9981561942087861 \\
164 & 0.9981561943425054 \\
165 & 0.998196252736138 \\
166 & 0.9982329232213443 \\
167 & 0.9982329233429073 \\
168 & 0.9982329233429073 \\
169 & 0.9982571867027132 \\
170 & 0.9982571867027132 \\
171 & 0.9983387685922098 \\
172 & 0.998338768692675 \\
173 & 0.998338768692675 \\
174 & 0.998338768692675 \\
175 & 0.998338768692675 \\
176 & 0.9983609113075549 \\
177 & 0.9983609113075549 \\
178 & 0.9983609113988869 \\
179 & 0.9983609113988869 \\
180 & 0.9983609113988869 \\
181 & 0.998360911398887 \\
182 & 0.9984156244669912 \\
183 & 0.9984156244669912 \\
184 & 0.9984156244669912 \\
185 & 0.9984156244669914 \\
186 & 0.9984156246192113 \\
187 & 0.9984156246192113 \\
188 & 0.9984156246192113 \\
189 & 0.9984156246192114 \\
190 & 0.9984357542691019 \\
191 & 0.9984357542691019 \\
192 & 0.9985293524894256 \\
193 & 0.9985293524894256 \\
194 & 0.9985293524894256 \\
195 & 0.9985293524894256 \\
196 & 0.9985293525586166 \\
197 & 0.9985293525586166 \\
198 & 0.9985694341602479 \\
199 & 0.9985694341602479 \\
200 & 0.9986448183592187 \\
201 & 0.9986448184164013 \\
202 & 0.9986448184164013 \\
203 & 0.9986448184164013 \\
204 & 0.9986448184164013 \\
205 & 0.9986448184164013 \\
206 & 0.9986812562360661 \\
207 & 0.998699485657408 \\
208 & 0.998699485657408 \\
209 & 0.998699485657408 \\
210 & 0.998699485657408 \\
211 & 0.998699485657408 \\
212 & 0.9986994857093922 \\
213 & 0.998716121783682 \\
214 & 0.998716121783682 \\
215 & 0.9987161218703222 \\
216 & 0.9987161218703222 \\
217 & 0.9987161218703222 \\
218 & 0.9987161218703222 \\
219 & 0.9987494381673743 \\
220 & 0.9987494381673743 \\
221 & 0.9987494381673745 \\
222 & 0.9987494381673742 \\
223 & 0.9987494381673743 \\
224 & 0.9987494381673744 \\
225 & 0.9987494381673743 \\
226 & 0.9987494381673744 \\
227 & 0.9987494381673743 \\
228 & 0.9987494381673744 \\
229 & 0.9987494381673743 \\
230 & 0.9987646209327019 \\
231 & 0.9987646209327018 \\
232 & 0.998764620932702 \\
233 & 0.9987646209720838 \\
234 & 0.9987646209720838 \\
235 & 0.9987646209720838 \\
236 & 0.9987646210078857 \\
237 & 0.9988056686843273 \\
238 & 0.9988056686843273 \\
239 & 0.998835970450361 \\
240 & 0.998835970450361 \\
241 & 0.998835970450361 \\
242 & 0.998835970450361 \\
243 & 0.9988359704799492 \\
244 & 0.9988359704799492 \\
245 & 0.9988359704799492 \\
246 & 0.9988359704799492 \\
247 & 0.9988359704799493 \\
248 & 0.9988359704799493 \\
249 & 0.9988359704799492 \\
250 & 0.9988359705068477 \\
251 & 0.9988359705068477 \\
252 & 0.9988359705068477 \\
253 & 0.9988631257315393 \\
254 & 0.9988631257315393 \\
255 & 0.9988631257315393 \\
256 & 0.9988631257315393 \\
257 & 0.9988631257559923 \\
258 & 0.9988631257559925 \\
259 & 0.9988631257782226 \\
260 & 0.9988631258152727 \\
261 & 0.9988631258152727 \\
262 & 0.9988631258321137 \\
263 & 0.9988631258321138 \\
264 & 0.9988631258474238 \\
265 & 0.9988631258474238 \\
266 & 0.9988631258474236 \\
267 & 0.9988631258474236 \\
268 & 0.9988631258474236 \\
269 & 0.9988631258613417 \\
270 & 0.998863125861342 \\
271 & 0.9988631258845387 \\
272 & 0.998876928398473 \\
273 & 0.998876928398473 \\
274 & 0.998876928398473 \\
275 & 0.998876928398473 \\
276 & 0.998876928398473 \\
277 & 0.998876928398473 \\
278 & 0.998876928409017 \\
279 & 0.998876928409017 \\
280 & 0.998876928409017 \\
281 & 0.998876928409017 \\
282 & 0.9989045558849897 \\
283 & 0.9989417915185373 \\
284 & 0.9989417915185373 \\
285 & 0.9989417915185373 \\
286 & 0.9989417915185373 \\
287 & 0.998966478086439 \\
288 & 0.998966478095153 \\
289 & 0.998966478095153 \\
290 & 0.998966478095153 \\
291 & 0.9989664780951532 \\
292 & 0.998966478095153 \\
293 & 0.9989664781030749 \\
294 & 0.9990004018026135 \\
295 & 0.999025444557561 \\
296 & 0.9990254445575611 \\
297 & 0.9990254445647629 \\
298 & 0.999048369896567 \\
299 & 0.999048369896567 \\
300 & 0.999048369896567 \\
301 & 0.999048369896567 \\
302 & 0.9990483699031141 \\
303 & 0.9990483699031141 \\
304 & 0.9990483699031141 \\
305 & 0.9990483699031141 \\
306 & 0.9990483699031141 \\
307 & 0.9990483699031142 \\
308 & 0.9990483699031141 \\
309 & 0.9990483699031141 \\
310 & 0.9990483699031142 \\
311 & 0.9990483699031142 \\
312 & 0.9990483699031141 \\
313 & 0.9990791431460581 \\
314 & 0.99907914315201 \\
315 & 0.9990791431574207 \\
316 & 0.9990791431574206 \\
317 & 0.9990791431574207 \\
318 & 0.9990791431623397 \\
319 & 0.9990791431623396 \\
320 & 0.999101492789762 \\
321 & 0.999101492789762 \\
322 & 0.9991503236591537 \\
323 & 0.9991503236591537 \\
324 & 0.9991503236591537 \\
325 & 0.9991503236591537 \\
326 & 0.9991503236591537 \\
327 & 0.9991503236591537 \\
328 & 0.9991503236636254 \\
329 & 0.9991503236636254 \\
330 & 0.9991503236636256 \\
331 & 0.9991503236636256 \\
332 & 0.9991628574203087 \\
333 & 0.9991742644566179 \\
334 & 0.9991742644566179 \\
335 & 0.9991742644566179 \\
336 & 0.9991742644566179 \\
337 & 0.9991742644566179 \\
338 & 0.999174264456618 \\
339 & 0.9991934442782914 \\
340 & 0.9991934442782914 \\
341 & 0.9992188767104765 \\
342 & 0.9992188767104765 \\
343 & 0.9992188767104764 \\
344 & 0.9992188767104764 \\
345 & 0.9992188767104764 \\
346 & 0.9992292467434849 \\
347 & 0.9992292467434848 \\
348 & 0.9992292467434848 \\
349 & 0.99922924674755 \\
350 & 0.99922924674755 \\
351 & 0.99922924674755 \\
352 & 0.99922924674755 \\
353 & 0.99922924674755 \\
354 & 0.9992491990233815 \\
355 & 0.9992491990233815 \\
356 & 0.9992491990233815 \\
357 & 0.9992491990233815 \\
358 & 0.9992491990233815 \\
359 & 0.9992491990233815 \\
360 & 0.9992491990233815 \\
361 & 0.9992491990233815 \\
362 & 0.9992491990233815 \\
363 & 0.9992491990270771 \\
364 & 0.9992491990304367 \\
365 & 0.9992491990304367 \\
366 & 0.9992491990304367 \\
367 & 0.9992491990304367 \\
368 & 0.9992491990304367 \\
369 & 0.9992723693736774 \\
370 & 0.9992723693736774 \\
371 & 0.9992723693736774 \\
372 & 0.9992723693736774 \\
373 & 0.9992723693736774 \\
374 & 0.9992723693736774 \\
375 & 0.9992723693736774 \\
376 & 0.9992723693736774 \\
377 & 0.9992723693736775 \\
378 & 0.9992723693767318 \\
379 & 0.9992723693795083 \\
380 & 0.9992723693795083 \\
381 & 0.9992723693795083 \\
382 & 0.9992723693795084 \\
383 & 0.9992723693795084 \\
384 & 0.9992723693795084 \\
385 & 0.9992723693795084 \\
386 & 0.9992723693795083 \\
387 & 0.9992723693795083 \\
388 & 0.9992723693795083 \\
389 & 0.9992723693795083 \\
390 & 0.9992723693795083 \\
391 & 0.999289887931501 \\
392 & 0.9992898879340252 \\
393 & 0.9993080263708062 \\
394 & 0.9993080263708062 \\
395 & 0.9993080263708062 \\
396 & 0.9993239977365679 \\
397 & 0.9993239977365679 \\
398 & 0.9993239977384801 \\
399 & 0.9993239977384801 \\
400 & 0.9993239977384801 \\
401 & 0.9993239977384801 \\
402 & 0.9993404872226384 \\
403 & 0.9993404872226384 \\
404 & 0.9993404872226385 \\
405 & 0.9993404872226385 \\
406 & 0.9993404872226384 \\
407 & 0.9993404872226384 \\
408 & 0.9993404872226385 \\
409 & 0.9993404872226384 \\
410 & 0.9993404872226384 \\
411 & 0.9993404872226384 \\
412 & 0.9993404872226385 \\
413 & 0.9993404872226385 \\
414 & 0.9993404872226384 \\
415 & 0.9993404872226384 \\
416 & 0.9993404872226384 \\
417 & 0.9993404872226384 \\
418 & 0.9993404872243766 \\
419 & 0.9993404872243767 \\
420 & 0.9993615162692007 \\
421 & 0.9993615162692007 \\
422 & 0.9993615162692007 \\
423 & 0.9993766726685991 \\
424 & 0.9993766726685991 \\
425 & 0.9993766726685991 \\
426 & 0.9993766726685991 \\
427 & 0.9993766726701795 \\
428 & 0.9993766726701795 \\
429 & 0.9993766726701795 \\
430 & 0.9993766726701795 \\
431 & 0.9993766726701795 \\
432 & 0.9993766726701795 \\
433 & 0.9993766726701795 \\
434 & 0.9993766726701795 \\
435 & 0.9993766726701795 \\
436 & 0.9993766726701795 \\
437 & 0.9993766726701795 \\
438 & 0.9993766726701795 \\
439 & 0.9993766726701795 \\
440 & 0.9993958873414234 \\
441 & 0.9993958873414234 \\
442 & 0.9993958873414234 \\
443 & 0.9993958873414234 \\
444 & 0.9993958873414234 \\
445 & 0.9994052480618396 \\
446 & 0.9994195602447279 \\
447 & 0.9994195602447279 \\
448 & 0.9994195602447279 \\
449 & 0.9994195602447279 \\
450 & 0.9994195602447279 \\
451 & 0.9994195602447279 \\
452 & 0.9994195602447279 \\
453 & 0.9994195602447279 \\
454 & 0.9994195602447279 \\
455 & 0.9994195602447279 \\
456 & 0.9994195602447279 \\
457 & 0.9994370281276769 \\
458 & 0.9994370281276769 \\
459 & 0.9994370281276768 \\
460 & 0.9994370281276768 \\
461 & 0.9994370281276768 \\
462 & 0.9994370281276768 \\
463 & 0.9994370281276769 \\
464 & 0.9994370281276768 \\
465 & 0.9994370281276768 \\
466 & 0.9994370281276769 \\
467 & 0.9994501900750793 \\
468 & 0.9994501900750793 \\
469 & 0.9994501900750793 \\
470 & 0.999450190076516 \\
471 & 0.999450190076516 \\
472 & 0.999450190076516 \\
473 & 0.999450190076516 \\
474 & 0.999450190076516 \\
475 & 0.9994622065718848 \\
476 & 0.9994622065718848 \\
477 & 0.9994622065718848 \\
478 & 0.9994622065718848 \\
479 & 0.9994622065718848 \\
480 & 0.9994622065718848 \\
481 & 0.9994622065718848 \\
482 & 0.9994622065718848 \\
483 & 0.9994622065718848 \\
484 & 0.9994622065718848 \\
485 & 0.9994622065718848 \\
486 & 0.9994622065718848 \\
487 & 0.9994622065718848 \\
488 & 0.999478052348717 \\
489 & 0.9994780523500232 \\
490 & 0.9994780523500231 \\
491 & 0.9994780523500231 \\
492 & 0.9994780523500231 \\
493 & 0.9994780523500231 \\
494 & 0.9994780523500231 \\
495 & 0.9994925112772477 \\
496 & 0.9994925112772476 \\
497 & 0.9994925112772477 \\
498 & 0.9994925112772477 \\
499 & 0.9994925112772477 \\
500 & 0.9994925112772477 \\
501 & 0.9994925112772477 \\
502 & 0.9994925112772477 \\
503 & 0.9994925112772477 \\
504 & 0.9994925112772478 \\
505 & 0.9994925112772476 \\
506 & 0.9995034043487122 \\
507 & 0.9995034043487122 \\
508 & 0.9995034043497917 \\
509 & 0.9995034043497917 \\
510 & 0.9995034043497917 \\
511 & 0.9995034043497917 \\
512 & 0.9995034043497917 \\
513 & 0.9995034043497917 \\
514 & 0.9995034043497917 \\
515 & 0.9995034043497918 \\
516 & 0.9995034043497918 \\
517 & 0.9995034043497918 \\
518 & 0.9995034043497919 \\
519 & 0.9995034043497918 \\
520 & 0.9995034043497917 \\
521 & 0.999503404350773 \\
522 & 0.999503404350773 \\
523 & 0.999503404350773 \\
524 & 0.999503404350773 \\
525 & 0.999503404350773 \\
526 & 0.999503404350773 \\
527 & 0.999503404350773 \\
528 & 0.999503404350773 \\
529 & 0.999503404350773 \\
530 & 0.9995034043507731 \\
531 & 0.999503404350773 \\
532 & 0.999503404350773 \\
533 & 0.999503404350773 \\
534 & 0.999503404350773 \\
535 & 0.999503404350773 \\
536 & 0.999503404350773 \\
537 & 0.999503404350773 \\
538 & 0.999503404350773 \\
539 & 0.999503404350773 \\
540 & 0.999503404350773 \\
541 & 0.999503404350773 \\
542 & 0.999503404350773 \\
543 & 0.999503404350773 \\
544 & 0.999503404350773 \\
545 & 0.999503404350773 \\
546 & 0.9995034043507729 \\
547 & 0.9995034043516651 \\
548 & 0.999503404352476 \\
549 & 0.999503404352476 \\
550 & 0.9995034043532133 \\
551 & 0.9995034043532133 \\
552 & 0.9995034043532133 \\
553 & 0.9995034043532133 \\
554 & 0.9995034043532133 \\
555 & 0.9995034043538835 \\
556 & 0.9995034043538835 \\
557 & 0.9995170150548601 \\
558 & 0.9995170150554694 \\
559 & 0.9995170150554694 \\
560 & 0.9995170150554694 \\
561 & 0.9995170150554694 \\
562 & 0.9995170150554694 \\
563 & 0.9995170150564849 \\
564 & 0.9995170150564849 \\
565 & 0.999517015056485 \\
566 & 0.999517015056485 \\
567 & 0.9995170150564849 \\
568 & 0.9995170150564849 \\
569 & 0.9995170150569465 \\
570 & 0.9995170150569465 \\
571 & 0.9995170150569466 \\
572 & 0.9995170150569465 \\
573 & 0.9995170150569465 \\
574 & 0.9995170150569465 \\
575 & 0.9995170150569465 \\
576 & 0.9995170150569465 \\
577 & 0.9995170150569465 \\
578 & 0.9995170150569465 \\
579 & 0.9995170150569465 \\
580 & 0.9995170150569465 \\
581 & 0.9995170150569465 \\
582 & 0.9995170150569465 \\
583 & 0.9995170150573661 \\
584 & 0.9995170150573661 \\
585 & 0.9995170150573661 \\
586 & 0.9995170150577476 \\
587 & 0.9995170150577476 \\
588 & 0.9995170150577476 \\
589 & 0.9995170150577476 \\
590 & 0.9995170150577476 \\
591 & 0.9995170150577476 \\
592 & 0.9995170150577476 \\
593 & 0.9995170150577476 \\
594 & 0.9995170150580944 \\
595 & 0.9995170150580944 \\
596 & 0.9995170150580944 \\
597 & 0.9995170150580944 \\
598 & 0.9995170150580944 \\
599 & 0.9995170150580944 \\
600 & 0.9995170150586724 \\
601 & 0.9995170150586724 \\
602 & 0.9995170150586724 \\
603 & 0.9995170150586724 \\
604 & 0.9995170150586724 \\
605 & 0.9995170150586724 \\
606 & 0.9995170150586724 \\
607 & 0.9995269460465807 \\
608 & 0.9995400940126237 \\
609 & 0.9995400940126237 \\
610 & 0.9995400940128863 \\
611 & 0.9995400940128863 \\
612 & 0.9995400940128863 \\
613 & 0.9995400940128863 \\
614 & 0.9995400940128863 \\
615 & 0.9995400940128863 \\
616 & 0.9995400940128863 \\
617 & 0.9995400940128863 \\
618 & 0.9995400940128863 \\
619 & 0.9995400940128863 \\
620 & 0.9995400940128863 \\
621 & 0.9995400940128863 \\
622 & 0.9995400940128863 \\
623 & 0.9995400940131253 \\
624 & 0.9995400940131253 \\
625 & 0.9995400940131253 \\
626 & 0.9995400940131253 \\
627 & 0.9995400940131253 \\
628 & 0.9995400940131253 \\
629 & 0.9995400940131253 \\
630 & 0.9995400940131253 \\
631 & 0.9995400940131253 \\
632 & 0.9995400940131253 \\
633 & 0.9995400940131253 \\
634 & 0.9995520723419532 \\
635 & 0.9995520723419532 \\
636 & 0.9995520723419532 \\
637 & 0.9995520723419532 \\
638 & 0.9995629617317968 \\
639 & 0.9995728611771091 \\
640 & 0.9995728611771091 \\
641 & 0.9995728611773262 \\
642 & 0.9995728611773262 \\
643 & 0.9995728611773262 \\
644 & 0.9995728611773262 \\
645 & 0.9995728611773262 \\
646 & 0.9995728611773262 \\
647 & 0.9995728611773262 \\
648 & 0.9995728611773262 \\
649 & 0.9995728611773262 \\
650 & 0.9995728611773262 \\
651 & 0.9995728611773262 \\
652 & 0.9995728611773262 \\
653 & 0.9995852345418504 \\
654 & 0.9995852345418504 \\
655 & 0.9995852345418506 \\
656 & 0.9995852345418504 \\
657 & 0.9995852345418504 \\
658 & 0.9995852345418504 \\
659 & 0.9995852345418504 \\
660 & 0.9995852345418504 \\
661 & 0.9995852345418504 \\
662 & 0.9995852345418504 \\
663 & 0.9995852345418504 \\
664 & 0.9995852345418504 \\
665 & 0.9995852345418504 \\
666 & 0.9995852345418504 \\
667 & 0.9995852345418504 \\
668 & 0.9995852345418504 \\
669 & 0.9995852345418504 \\
670 & 0.9995852345418504 \\
671 & 0.9995852345418504 \\
672 & 0.9995852345418504 \\
673 & 0.9995852345418504 \\
674 & 0.9995852345418504 \\
675 & 0.9995852345418504 \\
676 & 0.9995852345418504 \\
677 & 0.9995852345418504 \\
678 & 0.9995852345418504 \\
679 & 0.9995852345418504 \\
680 & 0.9995852345418504 \\
681 & 0.9995852345418504 \\
682 & 0.9995852345418504 \\
683 & 0.9995852345418504 \\
684 & 0.9995852345420478 \\
685 & 0.9995852345420478 \\
686 & 0.9995852345420478 \\
687 & 0.9995852345420478 \\
688 & 0.9995852345420478 \\
689 & 0.9995852345420478 \\
690 & 0.9995852345420478 \\
691 & 0.9995852345420478 \\
692 & 0.9995852345420478 \\
693 & 0.9995852345420478 \\
694 & 0.9995852345420478 \\
695 & 0.9995852345420478 \\
696 & 0.9995852345420478 \\
697 & 0.9995852345420478 \\
698 & 0.9995942340377864 \\
699 & 0.9995942340377864 \\
700 & 0.9995942340377864 \\
701 & 0.9995942340377864 \\
702 & 0.9995942340377864 \\
703 & 0.9995942340377864 \\
704 & 0.9995942340377864 \\
705 & 0.9996032365761869 \\
706 & 0.9996032365761869 \\
707 & 0.9996032365761869 \\
708 & 0.9996032365761869 \\
709 & 0.9996032365761869 \\
710 & 0.9996032365761869 \\
711 & 0.9996032365761869 \\
712 & 0.9996032365763663 \\
713 & 0.9996032365763663 \\
714 & 0.9996032365763663 \\
715 & 0.9996032365763663 \\
716 & 0.9996032365763663 \\
717 & 0.9996032365763663 \\
718 & 0.9996032365763663 \\
719 & 0.9996032365763663 \\
720 & 0.9996032365763663 \\
721 & 0.9996032365763663 \\
722 & 0.9996032365763663 \\
723 & 0.9996032365763663 \\
724 & 0.9996032365765294 \\
725 & 0.9996032365765294 \\
726 & 0.9996032365765295 \\
727 & 0.9996032365765294 \\
728 & 0.9996032365765294 \\
729 & 0.9996032365765294 \\
730 & 0.9996032365765294 \\
731 & 0.9996032365765294 \\
732 & 0.9996032365765294 \\
733 & 0.9996144850897334 \\
734 & 0.9996144850897333 \\
735 & 0.9996144850897333 \\
736 & 0.9996144850897333 \\
737 & 0.9996144850897333 \\
738 & 0.9996144850897333 \\
739 & 0.9996144850897333 \\
740 & 0.9996144850897333 \\
741 & 0.9996144850897333 \\
742 & 0.9996144850897333 \\
743 & 0.9996144850897333 \\
744 & 0.9996144850897333 \\
745 & 0.9996144850897333 \\
746 & 0.9996144850897333 \\
747 & 0.9996144850897333 \\
748 & 0.9996144850897333 \\
749 & 0.9996144850897333 \\
750 & 0.9996144850897334 \\
751 & 0.9996144850897334 \\
752 & 0.9996144850897333 \\
753 & 0.9996144850897333 \\
754 & 0.9996144850897333 \\
755 & 0.9996144850897333 \\
756 & 0.9996144850897333 \\
757 & 0.9996144850897333 \\
758 & 0.9996144850897333 \\
759 & 0.9996144850897333 \\
760 & 0.9996144850897333 \\
761 & 0.9996144850897333 \\
762 & 0.9996144850897333 \\
763 & 0.9996144850897333 \\
764 & 0.9996144850897333 \\
765 & 0.9996144850897333 \\
766 & 0.9996144850897333 \\
767 & 0.9996144850897333 \\
768 & 0.9996144850897333 \\
769 & 0.9996144850898815 \\
770 & 0.9996144850898815 \\
771 & 0.9996144850898815 \\
772 & 0.9996144850898815 \\
773 & 0.9996144850898815 \\
774 & 0.9996247110109759 \\
775 & 0.9996247110109759 \\
776 & 0.9996328951367947 \\
777 & 0.9996328951367947 \\
778 & 0.9996328951367948 \\
779 & 0.9996328951367947 \\
780 & 0.9996328951367947 \\
781 & 0.9996328951367947 \\
782 & 0.9996328951367945 \\
783 & 0.9996328951367945 \\
784 & 0.9996328951367945 \\
785 & 0.9996328951367945 \\
786 & 0.9996410772086335 \\
787 & 0.9996410772086335 \\
788 & 0.9996410772086335 \\
789 & 0.9996410772086335 \\
790 & 0.9996495318997575 \\
791 & 0.9996495318997575 \\
792 & 0.9996495318997575 \\
793 & 0.9996495318997575 \\
794 & 0.9996495318997575 \\
795 & 0.9996495318997575 \\
796 & 0.9996495318997575 \\
797 & 0.9996495318997575 \\
798 & 0.9996495318997575 \\
799 & 0.9996495318997575 \\
800 & 0.9996495318997575 \\
801 & 0.9996495318997575 \\
802 & 0.9996495318997575 \\
803 & 0.9996495318997575 \\
804 & 0.9996495318997575 \\
805 & 0.9996495318997575 \\
806 & 0.9996495318997576 \\
807 & 0.9996495318997575 \\
808 & 0.9996495318997575 \\
809 & 0.9996495318997575 \\
810 & 0.9996572186299398 \\
811 & 0.9996572186299398 \\
812 & 0.9996572186299398 \\
813 & 0.9996665149218439 \\
814 & 0.9996665149218439 \\
815 & 0.9996665149218439 \\
816 & 0.9996665149218439 \\
817 & 0.9996665149219787 \\
818 & 0.9996665149219787 \\
819 & 0.9996735028585083 \\
820 & 0.9996735028585083 \\
821 & 0.9996735028585083 \\
822 & 0.9996735028585083 \\
823 & 0.9996735028585083 \\
824 & 0.9996735028585083 \\
825 & 0.9996735028585082 \\
826 & 0.9996735028586308 \\
827 & 0.9996735028586308 \\
828 & 0.9996735028586308 \\
829 & 0.9996735028586308 \\
830 & 0.9996735028586308 \\
831 & 0.9996735028586308 \\
832 & 0.9996735028586308 \\
833 & 0.9996735028586308 \\
834 & 0.9996735028586308 \\
835 & 0.9996735028586308 \\
836 & 0.9996735028586308 \\
837 & 0.9996735028586308 \\
838 & 0.9996735028586308 \\
839 & 0.9996735028586309 \\
840 & 0.9996735028587423 \\
841 & 0.9996735028587422 \\
842 & 0.9996735028587422 \\
843 & 0.9996735028587422 \\
844 & 0.9996735028587422 \\
845 & 0.9996735028587422 \\
846 & 0.9996735028587422 \\
847 & 0.9996735028587422 \\
848 & 0.9996735028587422 \\
849 & 0.9996735028587422 \\
850 & 0.9996735028587423 \\
851 & 0.9996735028587423 \\
852 & 0.9996735028587422 \\
853 & 0.9996809429731228 \\
854 & 0.9996809429731228 \\
855 & 0.9996809429731228 \\
856 & 0.9996809429731228 \\
857 & 0.9996809429731228 \\
858 & 0.9996809429731228 \\
859 & 0.9996809429731228 \\
860 & 0.9996809429731228 \\
861 & 0.9996809429731228 \\
862 & 0.9996809429731228 \\
863 & 0.9996809429731228 \\
864 & 0.9996809429732242 \\
865 & 0.9996809429732242 \\
866 & 0.9996809429732242 \\
867 & 0.9996809429732241 \\
868 & 0.9996809429732241 \\
869 & 0.9996809429732242 \\
870 & 0.999680942973224 \\
871 & 0.9996809429732242 \\
872 & 0.9996809429732241 \\
873 & 0.9996809429732241 \\
874 & 0.9996809429732241 \\
875 & 0.9996809429732241 \\
876 & 0.9996809429732241 \\
877 & 0.9996809429732241 \\
878 & 0.9996809429732241 \\
879 & 0.9996809429732241 \\
880 & 0.9996809429732241 \\
881 & 0.9996809429732241 \\
882 & 0.9996809429732241 \\
883 & 0.9996809429732241 \\
884 & 0.9996809429732241 \\
885 & 0.9996809429732241 \\
886 & 0.9996809429732241 \\
887 & 0.9996809429732242 \\
888 & 0.9996809429732242 \\
889 & 0.9996809429732241 \\
890 & 0.9996809429732241 \\
891 & 0.9996809429732241 \\
892 & 0.9996883805730079 \\
893 & 0.9996883805730079 \\
894 & 0.9996883805730079 \\
895 & 0.9996883805730079 \\
896 & 0.9996883805730079 \\
897 & 0.9996883805730079 \\
898 & 0.9996883805730079 \\
899 & 0.9996883805730079 \\
900 & 0.999688380573008 \\
901 & 0.999695144313354 \\
902 & 0.999695144313354 \\
903 & 0.9996951443133539 \\
904 & 0.9996951443133539 \\
905 & 0.9996951443133539 \\
906 & 0.9996951443133539 \\
907 & 0.9996951443133539 \\
908 & 0.9996951443133539 \\
909 & 0.9996951443133539 \\
910 & 0.9996951443133539 \\
911 & 0.9996951443133539 \\
912 & 0.9996951443133539 \\
913 & 0.9996951443133539 \\
914 & 0.9996951443133539 \\
915 & 0.9996951443133539 \\
916 & 0.9996951443133539 \\
917 & 0.9996951443133539 \\
918 & 0.9996951443133539 \\
919 & 0.9996951443133539 \\
920 & 0.9996951443133539 \\
921 & 0.9996951443133539 \\
922 & 0.9996951443133539 \\
923 & 0.9996951443133539 \\
924 & 0.999695144313354 \\
925 & 0.999695144313354 \\
926 & 0.9996951443133539 \\
927 & 0.999695144313446 \\
928 & 0.9996951443134459 \\
929 & 0.999695144313446 \\
930 & 0.999695144313446 \\
931 & 0.999695144313446 \\
932 & 0.999695144313446 \\
933 & 0.999695144313446 \\
934 & 0.999695144313446 \\
935 & 0.999695144313446 \\
936 & 0.999695144313446 \\
937 & 0.999695144313446 \\
938 & 0.999695144313446 \\
939 & 0.999695144313446 \\
940 & 0.9996951443134461 \\
941 & 0.9996951443135297 \\
942 & 0.9996951443135297 \\
943 & 0.9996951443135297 \\
944 & 0.9996951443135297 \\
945 & 0.9996951443135297 \\
946 & 0.9996951443135297 \\
947 & 0.9996951443135297 \\
948 & 0.9996951443135297 \\
949 & 0.9996951443135297 \\
950 & 0.9996951443135297 \\
951 & 0.9996951443135297 \\
952 & 0.9996951443135297 \\
953 & 0.9996951443135297 \\
954 & 0.9996951443136058 \\
955 & 0.9996951443136058 \\
956 & 0.9996951443136058 \\
957 & 0.9996951443136058 \\
958 & 0.9996951443136058 \\
959 & 0.999695144313675 \\
960 & 0.999695144313675 \\
961 & 0.999695144313675 \\
962 & 0.999695144313675 \\
963 & 0.9997014969832473 \\
964 & 0.9997014969832473 \\
965 & 0.9997014969832473 \\
966 & 0.9997014969832473 \\
967 & 0.9997014969832474 \\
968 & 0.9997014969832473 \\
969 & 0.9997014969832473 \\
970 & 0.9997014969832473 \\
971 & 0.9997014969832473 \\
972 & 0.9997014969832473 \\
973 & 0.9997014969832473 \\
974 & 0.9997014969832473 \\
975 & 0.9997014969832473 \\
976 & 0.9997014969832474 \\
977 & 0.9997014969832473 \\
978 & 0.9997014969832473 \\
979 & 0.9997014969832473 \\
980 & 0.9997014969832473 \\
981 & 0.9997014969832473 \\
982 & 0.9997099481577054 \\
983 & 0.9997099481577054 \\
984 & 0.999715723311862 \\
985 & 0.999715723311862 \\
986 & 0.999715723311862 \\
987 & 0.999715723311862 \\
988 & 0.999715723311862 \\
989 & 0.999715723311862 \\
990 & 0.999715723311862 \\
991 & 0.999715723311862 \\
992 & 0.999715723311862 \\
993 & 0.999715723311862 \\
994 & 0.999715723311862 \\
995 & 0.999715723311862 \\
996 & 0.999715723311862 \\
997 & 0.999715723311862 \\
998 & 0.999715723311862 \\
999 & 0.999715723311862 \\
}\expectedgaindeci

\pgfplotstableread[row sep=\\,col sep=&]{
values & P \\
0 & 0.7318810568515274 \\
1 & 0.8064409137923557 \\
2 & 0.8318360754136418 \\
3 & 0.8497881829488712 \\
4 & 0.8603903047663723 \\
5 & 0.8672001571481214 \\
6 & 0.8725972992541466 \\
7 & 0.8773463938509242 \\
8 & 0.8808405793653348 \\
9 & 0.88600897735109 \\
10 & 0.8890039067192418 \\
11 & 0.8920505858546528 \\
12 & 0.894812559193665 \\
13 & 0.8964379605958694 \\
14 & 0.8979091386004169 \\
15 & 0.9001831566365857 \\
16 & 0.9014211544161178 \\
17 & 0.9028202824799675 \\
18 & 0.9050786045206501 \\
19 & 0.9062927636355005 \\
20 & 0.907490633939971 \\
21 & 0.9098004412660313 \\
22 & 0.9108421178460887 \\
23 & 0.9120617239970098 \\
24 & 0.9131677028037167 \\
25 & 0.9149735365997546 \\
26 & 0.916131938184022 \\
27 & 0.9168175911717373 \\
28 & 0.9176505055048291 \\
29 & 0.9181085783840678 \\
30 & 0.9191631646712053 \\
31 & 0.9199436409872732 \\
32 & 0.920535081098255 \\
33 & 0.9211401172123324 \\
34 & 0.9219730515052169 \\
35 & 0.9222961269288554 \\
36 & 0.9227191221926258 \\
37 & 0.9235469778174484 \\
38 & 0.9244792496129794 \\
39 & 0.9247559110692608 \\
40 & 0.9250504338349075 \\
41 & 0.9258131003611296 \\
42 & 0.9263435072757619 \\
43 & 0.9265942001221645 \\
44 & 0.9274603546671744 \\
45 & 0.9278080444088647 \\
46 & 0.9282657401434875 \\
47 & 0.9286273643109493 \\
48 & 0.9289127315682828 \\
49 & 0.9296027611629202 \\
50 & 0.9310535932511378 \\
51 & 0.9317553386605736 \\
52 & 0.9321681133472998 \\
53 & 0.9328317274566195 \\
54 & 0.9330266617406588 \\
55 & 0.9333704806808549 \\
56 & 0.9336624584745666 \\
57 & 0.9337354461987574 \\
58 & 0.934125453074106 \\
59 & 0.9377300943913258 \\
60 & 0.9383313350846472 \\
61 & 0.9384989361275504 \\
62 & 0.9388510143338802 \\
63 & 0.9396269525560331 \\
64 & 0.9399411496151912 \\
65 & 0.9399959793979537 \\
66 & 0.9402987805397384 \\
67 & 0.9407866996122884 \\
68 & 0.9416958510418566 \\
69 & 0.9421002519483714 \\
70 & 0.942737842423203 \\
71 & 0.9430108842558264 \\
72 & 0.943547975885601 \\
73 & 0.9438564527560496 \\
74 & 0.9439984901702071 \\
75 & 0.944759165143782 \\
76 & 0.9450164360193217 \\
77 & 0.9451943956596859 \\
78 & 0.9455121769070136 \\
79 & 0.9461434219240871 \\
80 & 0.9469071811407265 \\
81 & 0.9473936928152475 \\
82 & 0.947564123399667 \\
83 & 0.9480275901344433 \\
84 & 0.9482095988695031 \\
85 & 0.9483149603088457 \\
86 & 0.9485445424273959 \\
87 & 0.9486198157432828 \\
88 & 0.9486726964547872 \\
89 & 0.9486978216549063 \\
90 & 0.9489173062753641 \\
91 & 0.9493561439188403 \\
92 & 0.949544663741213 \\
93 & 0.9496269681541051 \\
94 & 0.9496280837452212 \\
95 & 0.9496486164281582 \\
96 & 0.9497254987311714 \\
97 & 0.9498682515926005 \\
98 & 0.9502007501015447 \\
99 & 0.950268762558764 \\
100 & 0.9505800382051114 \\
101 & 0.9509057021025128 \\
102 & 0.9509072428541511 \\
103 & 0.9512810228793885 \\
104 & 0.9516744493504886 \\
105 & 0.9517058786073339 \\
106 & 0.9517507757169639 \\
107 & 0.9517943457937833 \\
108 & 0.9522598588884208 \\
109 & 0.9524992191709301 \\
110 & 0.9527728784441329 \\
111 & 0.9528125017266362 \\
112 & 0.9529798872628432 \\
113 & 0.953158601491117 \\
114 & 0.9531966363216375 \\
115 & 0.9533732849562436 \\
116 & 0.9535715306876715 \\
117 & 0.9539166204877162 \\
118 & 0.9540688432953258 \\
119 & 0.9540804161677163 \\
120 & 0.9540811712205785 \\
121 & 0.9541135673077051 \\
122 & 0.9543115997100166 \\
123 & 0.9546258741267449 \\
124 & 0.9550925527147477 \\
125 & 0.9550931626421973 \\
126 & 0.9552253310850994 \\
127 & 0.9555940629945066 \\
128 & 0.955693187903285 \\
129 & 0.9559866926888493 \\
130 & 0.9560058137179411 \\
131 & 0.9561213459333882 \\
132 & 0.9563224126196873 \\
133 & 0.9564359217891589 \\
134 & 0.9566952602479861 \\
135 & 0.9572329436640797 \\
136 & 0.9572497178274364 \\
137 & 0.9572738861046431 \\
138 & 0.9575496504050509 \\
139 & 0.957648823762719 \\
140 & 0.9579629085911097 \\
141 & 0.9579704163462912 \\
142 & 0.9580608928237122 \\
143 & 0.9581650486003325 \\
144 & 0.9581862537995783 \\
145 & 0.9584564351703292 \\
146 & 0.9586031078229992 \\
147 & 0.9586163525176741 \\
148 & 0.9586356517957281 \\
149 & 0.9586420293906701 \\
150 & 0.9588222917013286 \\
151 & 0.9591505107207395 \\
152 & 0.9591566969406985 \\
153 & 0.9593846577881014 \\
154 & 0.9593847945306121 \\
155 & 0.9595266530482626 \\
156 & 0.9596225423496374 \\
157 & 0.9597929996219916 \\
158 & 0.9598046377884274 \\
159 & 0.9599450791882412 \\
160 & 0.9599564848373418 \\
161 & 0.9601289150676926 \\
162 & 0.9604451723135056 \\
163 & 0.9606195334368494 \\
164 & 0.9609139138491538 \\
165 & 0.9609935008598988 \\
166 & 0.9612047183858535 \\
167 & 0.9612149657518939 \\
168 & 0.9614291488800387 \\
169 & 0.9614341801682865 \\
170 & 0.9616412406739365 \\
171 & 0.9617846293904951 \\
172 & 0.961860937475459 \\
173 & 0.9619980115109495 \\
174 & 0.9621474009190475 \\
175 & 0.9624838780941946 \\
176 & 0.9625539221650419 \\
177 & 0.962571318316755 \\
178 & 0.9625756503922677 \\
179 & 0.9625756650749434 \\
180 & 0.9625757000336952 \\
181 & 0.9625841813091263 \\
182 & 0.9627703123124189 \\
183 & 0.9630953334628846 \\
184 & 0.9632239149944358 \\
185 & 0.9633598517342465 \\
186 & 0.9634349854204448 \\
187 & 0.9634426428711512 \\
188 & 0.9634501592414336 \\
189 & 0.9634538867177163 \\
190 & 0.963520525084037 \\
191 & 0.9635972942467603 \\
192 & 0.963727718853072 \\
193 & 0.9637999775552795 \\
194 & 0.964097103351927 \\
195 & 0.9641651359050967 \\
196 & 0.9642357629203449 \\
197 & 0.9642991703694673 \\
198 & 0.9643087774143678 \\
199 & 0.9644999419762318 \\
200 & 0.964627020191367 \\
201 & 0.9646359850015351 \\
202 & 0.9646360067480574 \\
203 & 0.9646360146654805 \\
204 & 0.9648201137640551 \\
205 & 0.9649434574296738 \\
206 & 0.964951987750533 \\
207 & 0.9652519297137087 \\
208 & 0.965257389809499 \\
209 & 0.9652627494159202 \\
210 & 0.9653816271667855 \\
211 & 0.9653893477294642 \\
212 & 0.9653893517794295 \\
213 & 0.9653944024468467 \\
214 & 0.9654064195288045 \\
215 & 0.9654111372056967 \\
216 & 0.9655875542861435 \\
217 & 0.9655898666108759 \\
218 & 0.9657078848816782 \\
219 & 0.9659378843318875 \\
220 & 0.9660502127866476 \\
221 & 0.9660566798840342 \\
222 & 0.9661165151176682 \\
223 & 0.9661715776756956 \\
224 & 0.96628965259465 \\
225 & 0.9664024533323513 \\
226 & 0.9664590195178632 \\
227 & 0.9665747252716979 \\
228 & 0.9665786732442846 \\
229 & 0.9665825413376089 \\
230 & 0.9668592931299995 \\
231 & 0.9668648161251711 \\
232 & 0.9669768331016664 \\
233 & 0.967086200135807 \\
234 & 0.967087954354399 \\
235 & 0.9673614273544717 \\
236 & 0.9673614310885746 \\
237 & 0.9674711051035154 \\
238 & 0.9674727915058116 \\
239 & 0.9675316518742155 \\
240 & 0.9675348265623628 \\
241 & 0.9676375540294063 \\
242 & 0.9677398364739241 \\
243 & 0.9677429193219227 \\
244 & 0.9677972601150199 \\
245 & 0.9680607254267778 \\
246 & 0.9682731590577951 \\
247 & 0.9682759787496403 \\
248 & 0.9683749744143334 \\
249 & 0.9684828785648765 \\
250 & 0.9684855091218731 \\
251 & 0.9685909899038537 \\
252 & 0.9686954251784006 \\
253 & 0.9688989509142564 \\
254 & 0.9690003381909469 \\
255 & 0.9690003403252737 \\
256 & 0.9690994739010598 \\
257 & 0.9691019510441967 \\
258 & 0.9691055580507804 \\
259 & 0.9691525379455556 \\
260 & 0.9692506898174432 \\
261 & 0.9693490608006129 \\
262 & 0.9694487461178573 \\
263 & 0.9695938877176178 \\
264 & 0.9696882088288014 \\
265 & 0.9696904106391367 \\
266 & 0.9696925691942069 \\
267 & 0.9696946847106088 \\
268 & 0.9697921391748957 \\
269 & 0.969794133095036 \\
270 & 0.9697951206250792 \\
271 & 0.9697951212130153 \\
272 & 0.9697960980748974 \\
273 & 0.9697960986513053 \\
274 & 0.9698885622501113 \\
275 & 0.9698885627883072 \\
276 & 0.9699355817740554 \\
277 & 0.9699374784836744 \\
278 & 0.9699867217978736 \\
279 & 0.9700353731215249 \\
280 & 0.9701269201148778 \\
281 & 0.970126921016936 \\
282 & 0.9701741717917819 \\
283 & 0.970333740035569 \\
284 & 0.9703362191224356 \\
285 & 0.970424678435054 \\
286 & 0.9706036248982776 \\
287 & 0.9707829090232322 \\
288 & 0.9707836811885157 \\
289 & 0.9707844453241191 \\
290 & 0.9707866699044806 \\
291 & 0.9709468172968118 \\
292 & 0.9709482291944577 \\
293 & 0.9709482296344146 \\
294 & 0.9710369116933193 \\
295 & 0.9711679298219672 \\
296 & 0.9712533284881023 \\
297 & 0.971340236913288 \\
298 & 0.9713834435964429 \\
299 & 0.9714255768304499 \\
300 & 0.9715127978491718 \\
301 & 0.9715134003795575 \\
302 & 0.9715157173753338 \\
303 & 0.9715567661972216 \\
304 & 0.9716366973788417 \\
305 & 0.9716765482748587 \\
306 & 0.9716776508384574 \\
307 & 0.9716781965658453 \\
308 & 0.9716792668008739 \\
309 & 0.9716797966183998 \\
310 & 0.9716803213593812 \\
311 & 0.9716808407305889 \\
312 & 0.9719153193560871 \\
313 & 0.972072152315185 \\
314 & 0.9720731121637131 \\
315 & 0.9720745100013753 \\
316 & 0.9720758670523316 \\
317 & 0.9720767539154701 \\
318 & 0.972161743430033 \\
319 & 0.9721998807240855 \\
320 & 0.97223807490956 \\
321 & 0.9722754607322178 \\
322 & 0.972441245470885 \\
323 & 0.9726334732461148 \\
324 & 0.9726347026473219 \\
325 & 0.9726347027638181 \\
326 & 0.9726347028780301 \\
327 & 0.9726347030443584 \\
328 & 0.9726347030992523 \\
329 & 0.9726355066205916 \\
330 & 0.9727888484101933 \\
331 & 0.9728712776582737 \\
332 & 0.97294321233087 \\
333 & 0.9729784450039596 \\
334 & 0.9730588995820673 \\
335 & 0.9730603703293393 \\
336 & 0.9730607344300873 \\
337 & 0.9730610949666941 \\
338 & 0.9731301784975382 \\
339 & 0.9732108863017016 \\
340 & 0.9732450860486151 \\
341 & 0.9733979286650082 \\
342 & 0.9734321261956692 \\
343 & 0.9734324593701414 \\
344 & 0.9734663150838392 \\
345 & 0.9734666415482259 \\
346 & 0.9735007864232955 \\
347 & 0.9735007864232955 \\
348 & 0.9735014079399952 \\
349 & 0.973613751837015 \\
350 & 0.9736140535021901 \\
351 & 0.9736471848556773 \\
352 & 0.9736480462959053 \\
353 & 0.973648882553218 \\
354 & 0.9737269709161237 \\
355 & 0.9737272469894025 \\
356 & 0.973727247106331 \\
357 & 0.9738045624773632 \\
358 & 0.9738053663622167 \\
359 & 0.9738053664147345 \\
360 & 0.9738378494906965 \\
361 & 0.973870011020626 \\
362 & 0.9738705260956498 \\
363 & 0.9739026139636507 \\
364 & 0.9739026140089784 \\
365 & 0.9739031090601231 \\
366 & 0.9739346243366377 \\
367 & 0.9739348670175428 \\
368 & 0.973935107295244 \\
369 & 0.973936031297126 \\
370 & 0.9739362600865492 \\
371 & 0.9739362601457625 \\
372 & 0.9739367086317261 \\
373 & 0.9740747944738272 \\
374 & 0.9742253272294091 \\
375 & 0.9742257540829674 \\
376 & 0.9743004781842178 \\
377 & 0.9743004781842178 \\
378 & 0.9745064946123116 \\
379 & 0.9745786220169576 \\
380 & 0.9745788271099171 \\
381 & 0.9745792292832524 \\
382 & 0.9745796235552563 \\
383 & 0.9747214394548515 \\
384 & 0.9747214394548515 \\
385 & 0.9747513910884764 \\
386 & 0.9747808548128607 \\
387 & 0.9747812337351831 \\
388 & 0.9747816052272684 \\
389 & 0.974810777230814 \\
390 & 0.9748107772688491 \\
391 & 0.9749509074019508 \\
392 & 0.9749510895390257 \\
393 & 0.9749803297365883 \\
394 & 0.9749806797839697 \\
395 & 0.9749808530993075 \\
396 & 0.9749810246680761 \\
397 & 0.9750097917345206 \\
398 & 0.9750790720793158 \\
399 & 0.9751354780619528 \\
400 & 0.975204398952935 \\
401 & 0.9752321578480624 \\
402 & 0.9752326333962158 \\
403 & 0.9752327903504097 \\
404 & 0.9753008609835156 \\
405 & 0.975300861019106 \\
406 & 0.9753283450619527 \\
407 & 0.9753286497553013 \\
408 & 0.9753288005837223 \\
409 & 0.9753563082355156 \\
410 & 0.9753833971273894 \\
411 & 0.9753833971519567 \\
412 & 0.9753835421090713 \\
413 & 0.9754106400361374 \\
414 & 0.9754373275989044 \\
415 & 0.9754636143355567 \\
416 & 0.9754895055556333 \\
417 & 0.9754896408325805 \\
418 & 0.9754897747691814 \\
419 & 0.9755156722014856 \\
420 & 0.9755410532458092 \\
421 & 0.9755411832715951 \\
422 & 0.9755663130222979 \\
423 & 0.9756588153487286 \\
424 & 0.9756588153687069 \\
425 & 0.975725392397703 \\
426 & 0.9757255173524836 \\
427 & 0.975725641084436 \\
428 & 0.975725641094988 \\
429 & 0.9757256410984705 \\
430 & 0.9757915589371938 \\
431 & 0.9757915589536129 \\
432 & 0.975791919288432 \\
433 & 0.9758573033995575 \\
434 & 0.9758575365966384 \\
435 & 0.9758575366053971 \\
436 & 0.9758575366111216 \\
437 & 0.9758577652236342 \\
438 & 0.9758578784185373 \\
439 & 0.9758578784284391 \\
440 & 0.9759069942024756 \\
441 & 0.9759312545483013 \\
442 & 0.9759312545597436 \\
443 & 0.9759312545664091 \\
444 & 0.9759313622833683 \\
445 & 0.9759552747211659 \\
446 & 0.9759554859202746 \\
447 & 0.9759792661579035 \\
448 & 0.9759792661654988 \\
449 & 0.9760028109473341 \\
450 & 0.9760030139271177 \\
451 & 0.9760032129269056 \\
452 & 0.9760033114434061 \\
453 & 0.9760267137782629 \\
454 & 0.9760267137833719 \\
455 & 0.9760269031666394 \\
456 & 0.9760269969258982 \\
457 & 0.9760501601277708 \\
458 & 0.9760502511325644 \\
459 & 0.9760502511368903 \\
460 & 0.9760504295732656 \\
461 & 0.9760504295760649 \\
462 & 0.9760733563744585 \\
463 & 0.976073611138613 \\
464 & 0.9760736952221932 \\
465 & 0.9760738600868856 \\
466 & 0.9760963866850961 \\
467 & 0.9760967037289 \\
468 & 0.9760967037312447 \\
469 & 0.9760967822105523 \\
470 & 0.9761191634671891 \\
471 & 0.9761191634683049 \\
472 & 0.9761191634704925 \\
473 & 0.9761192404053095 \\
474 & 0.9761193165752808 \\
475 & 0.9761196095318275 \\
476 & 0.976184301041482 \\
477 & 0.9761844432271494 \\
478 & 0.9761845136199963 \\
479 & 0.9762065963518772 \\
480 & 0.9762065963535559 \\
481 & 0.9762066660458513 \\
482 & 0.9762067350473093 \\
483 & 0.9762707825662686 \\
484 & 0.9762707825662685 \\
485 & 0.976270782567866 \\
486 & 0.9762707825694322 \\
487 & 0.9762927806144567 \\
488 & 0.976442445888502 \\
489 & 0.9764425122054787 \\
490 & 0.9764425778651066 \\
491 & 0.9764426428753138 \\
492 & 0.9764427072404647 \\
493 & 0.9764428334473048 \\
494 & 0.9764857042102532 \\
495 & 0.9764857042115511 \\
496 & 0.9764857042140475 \\
497 & 0.9764857666934958 \\
498 & 0.9764858285554213 \\
499 & 0.9766112779706806 \\
500 & 0.9766112779718009 \\
501 & 0.9766112779734323 \\
502 & 0.9766114581464611 \\
503 & 0.9766327407600267 \\
504 & 0.9766327996359719 \\
505 & 0.9766329150755909 \\
506 & 0.976633028252171 \\
507 & 0.9766331392091337 \\
508 & 0.9766331941385599 \\
509 & 0.976633194140815 \\
510 & 0.976633301845528 \\
511 & 0.9766334074383841 \\
512 & 0.9766335109599417 \\
513 & 0.9767817456468108 \\
514 & 0.9767818461336653 \\
515 & 0.9768440483224781 \\
516 & 0.9768441449081149 \\
517 & 0.9768441449089084 \\
518 & 0.9768441927239573 \\
519 & 0.9769056815075293 \\
520 & 0.9769665614908866 \\
521 & 0.9769666552458114 \\
522 & 0.9769667016594789 \\
523 & 0.9769878498613156 \\
524 & 0.9769878498616611 \\
525 & 0.9769878498620032 \\
526 & 0.9769878944769821 \\
527 & 0.9771072669936149 \\
528 & 0.9771072669945727 \\
529 & 0.9771281161381451 \\
530 & 0.9771872114438146 \\
531 & 0.9772078541593023 \\
532 & 0.9772078541595944 \\
533 & 0.977207897463516 \\
534 & 0.9772080235876661 \\
535 & 0.9772081060218439 \\
536 & 0.9772081868396679 \\
537 & 0.9772082268491445 \\
538 & 0.9772082268494041 \\
539 & 0.9772083807294903 \\
540 & 0.9772084188192359 \\
541 & 0.9772084188192358 \\
542 & 0.9772488089761064 \\
543 & 0.9772488466882294 \\
544 & 0.9772689516561716 \\
545 & 0.9772689516563998 \\
546 & 0.9772689516566259 \\
547 & 0.9772690234490445 \\
548 & 0.9772690589894236 \\
549 & 0.9772690589898376 \\
550 & 0.9773275691929915 \\
551 & 0.9773668105128627 \\
552 & 0.9773862884856265 \\
553 & 0.9775330389181462 \\
554 & 0.9775330389186978 \\
555 & 0.9775330727374454 \\
556 & 0.9775330727383124 \\
557 & 0.9775331712393285 \\
558 & 0.9775524205561197 \\
559 & 0.9775524842982256 \\
560 & 0.9775715108409427 \\
561 & 0.9775715733335145 \\
562 & 0.977590442431066 \\
563 & 0.9776091247057169 \\
564 & 0.9776092425153238 \\
565 & 0.9776093282990582 \\
566 & 0.9776093566107679 \\
567 & 0.977667369965296 \\
568 & 0.9776673699657188 \\
569 & 0.9776674500245263 \\
570 & 0.9776674500247998 \\
571 & 0.9776674500251982 \\
572 & 0.9777248850730509 \\
573 & 0.9777433520476269 \\
574 & 0.9778184764092055 \\
575 & 0.9778184764092054 \\
576 & 0.9778185262152065 \\
577 & 0.9778928811292124 \\
578 & 0.9778928811292124 \\
579 & 0.9778929759614167 \\
580 & 0.9779284722403638 \\
581 & 0.97792849571384 \\
582 & 0.9779284957142516 \\
583 & 0.9779285189554049 \\
584 & 0.9780017848318902 \\
585 & 0.9780569040896027 \\
586 & 0.9780569040899757 \\
587 & 0.9780569268729851 \\
588 & 0.9780743699764023 \\
589 & 0.978091618260693 \\
590 & 0.9780916182609541 \\
591 & 0.9780916826739533 \\
592 & 0.9780917452109247 \\
593 & 0.9780917860850097 \\
594 & 0.9780918063198425 \\
595 & 0.9780918459954128 \\
596 & 0.978091845995688 \\
597 & 0.9780918656370567 \\
598 & 0.9780918656371241 \\
599 & 0.9780918850840233 \\
600 & 0.9780918850843412 \\
601 & 0.9780919043388752 \\
602 & 0.978091923402648 \\
603 & 0.9781089978906392 \\
604 & 0.9781090349003709 \\
605 & 0.9781090711844217 \\
606 & 0.9781091067569085 \\
607 & 0.9781091416318958 \\
608 & 0.9781260625687981 \\
609 & 0.9781260794951717 \\
610 & 0.9781592075198908 \\
611 & 0.9781592075199952 \\
612 & 0.978159224112901 \\
613 & 0.9781592241130977 \\
614 & 0.9781756796473776 \\
615 & 0.9781757109309918 \\
616 & 0.9782078515502627 \\
617 & 0.9782078515504857 \\
618 & 0.9782237934182817 \\
619 & 0.9782237934183683 \\
620 & 0.978239547079631 \\
621 & 0.9783097629516602 \\
622 & 0.9783097629517371 \\
623 & 0.9783097629518853 \\
624 & 0.9783792541351799 \\
625 & 0.9784327669387445 \\
626 & 0.9784327669387446 \\
627 & 0.9784328094487215 \\
628 & 0.9784480997968972 \\
629 & 0.9784481138267046 \\
630 & 0.978448113826771 \\
631 & 0.9784481138268678 \\
632 & 0.9784481413360916 \\
633 & 0.9784481413361229 \\
634 & 0.9784481549545192 \\
635 & 0.9784481684381712 \\
636 & 0.9784481817882614 \\
637 & 0.9784481817883204 \\
638 & 0.9784481950063151 \\
639 & 0.9786086106909367 \\
640 & 0.978638345258188 \\
641 & 0.9786383819097402 \\
642 & 0.9786530891706342 \\
643 & 0.978667638838655 \\
644 & 0.9786676853625448 \\
645 & 0.9786676853626095 \\
646 & 0.9786676968784646 \\
647 & 0.9786676968785452 \\
648 & 0.9786677194585885 \\
649 & 0.9786677415958287 \\
650 & 0.9786677415959039 \\
651 & 0.9786677415959407 \\
652 & 0.9786677415959768 \\
653 & 0.9787207423579928 \\
654 & 0.9787207423579928 \\
655 & 0.9787875920522376 \\
656 & 0.9787875920522716 \\
657 & 0.9787876026921232 \\
658 & 0.9788018656741241 \\
659 & 0.9788159874384967 \\
660 & 0.9788159979730334 \\
661 & 0.9788679228071627 \\
662 & 0.9788819151822415 \\
663 & 0.9788819151822565 \\
664 & 0.9788819255092328 \\
665 & 0.9788819457580931 \\
666 & 0.9789333564849518 \\
667 & 0.9789333564849795 \\
668 & 0.9789333564850453 \\
669 & 0.9789333564850712 \\
670 & 0.9789472389350148 \\
671 & 0.9789472485737165 \\
672 & 0.9789981502834615 \\
673 & 0.9789981598267169 \\
674 & 0.9791119255988435 \\
675 & 0.9791254963368103 \\
676 & 0.9791255148637186 \\
677 & 0.9791389604096796 \\
678 & 0.9791389783934876 \\
679 & 0.979138978393508 \\
680 & 0.9791653331281692 \\
681 & 0.9791653331281985 \\
682 & 0.9791783930757919 \\
683 & 0.9791784017176959 \\
684 & 0.9791784102740541 \\
685 & 0.9791784270511649 \\
686 & 0.9791784353566954 \\
687 & 0.979204017091693 \\
688 & 0.9792040252335544 \\
689 & 0.9792040252335786 \\
690 & 0.9792538735162295 \\
691 & 0.9792538735162452 \\
692 & 0.9792538893226207 \\
693 & 0.9792539123415601 \\
694 & 0.9792539199385913 \\
695 & 0.9792665949890319 \\
696 & 0.9792665949890457 \\
697 & 0.979266602363371 \\
698 & 0.9792791444709235 \\
699 & 0.979279165736831 \\
700 & 0.9792915764360914 \\
701 & 0.9792915903353796 \\
702 & 0.9793038850372746 \\
703 & 0.9793038850372808 \\
704 & 0.9793160511967424 \\
705 & 0.979316058009454 \\
706 & 0.9793160713677178 \\
707 & 0.9793160713677295 \\
708 & 0.9793281170701662 \\
709 & 0.9793281170701776 \\
710 & 0.9793281170701941 \\
711 & 0.9793281301665295 \\
712 & 0.9793400630879369 \\
713 & 0.9793400758003518 \\
714 & 0.9793400820936451 \\
715 & 0.9793400820936631 \\
716 & 0.9793400820936675 \\
717 & 0.9793400944334167 \\
718 & 0.9793401065312101 \\
719 & 0.9793401065312186 \\
720 & 0.9793401065312312 \\
721 & 0.9793401125202438 \\
722 & 0.9793894592741195 \\
723 & 0.9793894652038386 \\
724 & 0.97938946520385 \\
725 & 0.9794012851852326 \\
726 & 0.9794012851852364 \\
727 & 0.9794012851852577 \\
728 & 0.9794012965841605 \\
729 & 0.9794129991996666 \\
730 & 0.9794734385874672 \\
731 & 0.979484905137616 \\
732 & 0.9795332795654971 \\
733 & 0.9796281368082153 \\
734 & 0.9796281421778932 \\
735 & 0.9796281628304598 \\
736 & 0.9796281729542756 \\
737 & 0.9796395255626488 \\
738 & 0.9796395305248168 \\
739 & 0.9796507706819881 \\
740 & 0.9796507940773908 \\
741 & 0.9796619180814913 \\
742 & 0.979719892925904 \\
743 & 0.9797199020097292 \\
744 & 0.9797199065066776 \\
745 & 0.9797199065066825 \\
746 & 0.9797199109591016 \\
747 & 0.9797308201847154 \\
748 & 0.9797308245494089 \\
749 & 0.9797522108655364 \\
750 & 0.9797522193390281 \\
751 & 0.9797522316790473 \\
752 & 0.9797522357516707 \\
753 & 0.9797522397839628 \\
754 & 0.9797628289221989 \\
755 & 0.9797628289222009 \\
756 & 0.9797628328750428 \\
757 & 0.979773313217488 \\
758 & 0.9797733209681507 \\
759 & 0.9797733285668415 \\
760 & 0.9798038575978173 \\
761 & 0.9798038649737585 \\
762 & 0.9798038722050697 \\
763 & 0.9798139536408766 \\
764 & 0.9798704266235655 \\
765 & 0.9798704266235708 \\
766 & 0.9798803125875374 \\
767 & 0.979880319203291 \\
768 & 0.9798803224784174 \\
769 & 0.9798901037823705 \\
770 & 0.9798901069929622 \\
771 & 0.9798997914205351 \\
772 & 0.9798997945678748 \\
773 & 0.9798997945678777 \\
774 & 0.9800527753821607 \\
775 & 0.9802135459367943 \\
776 & 0.980259580112252 \\
777 & 0.9802688845141343 \\
778 & 0.9803144629056847 \\
779 & 0.9803144659002655 \\
780 & 0.9803144659002667 \\
781 & 0.980323681144283 \\
782 & 0.9803688082056207 \\
783 & 0.9803688167558314 \\
784 & 0.9803688195776852 \\
785 & 0.9803779434400021 \\
786 & 0.9804226235584937 \\
787 & 0.9804316649587134 \\
788 & 0.9804406089395626 \\
789 & 0.9804406115470107 \\
790 & 0.9804581436005869 \\
791 & 0.980502381184436 \\
792 & 0.9805023887037454 \\
793 & 0.9805023983438855 \\
794 & 0.9805023983438886 \\
795 & 0.9805023983438896 \\
796 & 0.980502400730064 \\
797 & 0.9805110823270579 \\
798 & 0.980511084666217 \\
799 & 0.9805110892527988 \\
800 & 0.9805110915233862 \\
801 & 0.9805110915233913 \\
802 & 0.9805110915233939 \\
803 & 0.9805110937714991 \\
804 & 0.9805110959973442 \\
805 & 0.9805196892987732 \\
806 & 0.9805196892987755 \\
807 & 0.9805196892987785 \\
808 & 0.9805196892987794 \\
809 & 0.9805196892987807 \\
810 & 0.9805281975180172 \\
811 & 0.9805281975180195 \\
812 & 0.980536621497463 \\
813 & 0.9805366237012721 \\
814 & 0.9805366280224638 \\
815 & 0.9805366301616681 \\
816 & 0.9805366301616688 \\
817 & 0.980544972853398 \\
818 & 0.9807293388907857 \\
819 & 0.9807727083966258 \\
820 & 0.9807727083966264 \\
821 & 0.9807727123887308 \\
822 & 0.9807727123887319 \\
823 & 0.9807727143650222 \\
824 & 0.9807809762338672 \\
825 & 0.9807891543836754 \\
826 & 0.9807972496623448 \\
827 & 0.9807972515617087 \\
828 & 0.9807972534422661 \\
829 & 0.9808401874996119 \\
830 & 0.9808401911869761 \\
831 & 0.9808401930124053 \\
832 & 0.9808401948197601 \\
833 & 0.9808401948197606 \\
834 & 0.9808402000839002 \\
835 & 0.9808827107889107 \\
836 & 0.9808827125090492 \\
837 & 0.9809248005958517 \\
838 & 0.9809744870963149 \\
839 & 0.9809744887994235 \\
840 & 0.9809824279085456 \\
841 & 0.9809824279085464 \\
842 & 0.9809824279085472 \\
843 & 0.9809902883802123 \\
844 & 0.9809902883802145 \\
845 & 0.9809902915899665 \\
846 & 0.9809902915899679 \\
847 & 0.9809980725826357 \\
848 & 0.9810545866738164 \\
849 & 0.9810545866738166 \\
850 & 0.9810545882314934 \\
851 & 0.9810954400561419 \\
852 & 0.9811358889043232 \\
853 & 0.9811358918690569 \\
854 & 0.9811434432218835 \\
855 & 0.9811434432218844 \\
856 & 0.9811434446895744 \\
857 & 0.9811434475673974 \\
858 & 0.9811434475673974 \\
859 & 0.9811509269757491 \\
860 & 0.9811509283724795 \\
861 & 0.9811583323167242 \\
862 & 0.9813448834819049 \\
863 & 0.9813448834819054 \\
864 & 0.9813845323635381 \\
865 & 0.9813917904208311 \\
866 & 0.9813917930789738 \\
867 & 0.9813917943948861 \\
868 & 0.9813989818282147 \\
869 & 0.9813989831055516 \\
870 & 0.9814060968215013 \\
871 & 0.9814061005418993 \\
872 & 0.9814061017697534 \\
873 & 0.981406102985451 \\
874 & 0.9814061029854514 \\
875 & 0.981406104189112 \\
876 & 0.9814061041891123 \\
877 & 0.9814131498323491 \\
878 & 0.9814131521461911 \\
879 & 0.9814131532916578 \\
880 & 0.9814201279734248 \\
881 & 0.9814201279734255 \\
882 & 0.981420127973426 \\
883 & 0.9814201279734266 \\
884 & 0.9814201290963227 \\
885 & 0.9814593865265485 \\
886 & 0.9814593865265486 \\
887 & 0.9814593865265487 \\
888 & 0.9814593865265488 \\
889 & 0.9814593865265493 \\
890 & 0.98145938762732 \\
891 & 0.9814593887171918 \\
892 & 0.9814730623400192 \\
893 & 0.9814730623400193 \\
894 & 0.9814730623400193 \\
895 & 0.981473064477023 \\
896 & 0.9814798346551573 \\
897 & 0.9814798346551575 \\
898 & 0.9814798346551578 \\
899 & 0.9814798346551578 \\
900 & 0.9814865378018254 \\
901 & 0.9814865408526453 \\
902 & 0.9815254084946727 \\
903 & 0.9815254114566332 \\
904 & 0.9815638952480665 \\
905 & 0.9815705309898679 \\
906 & 0.9815705309898686 \\
907 & 0.9815705309898688 \\
908 & 0.9815705309898689 \\
909 & 0.9815705309898689 \\
910 & 0.9815705338370866 \\
911 & 0.9815771048181501 \\
912 & 0.9815771048181504 \\
913 & 0.9816152066140794 \\
914 & 0.9816152084565812 \\
915 & 0.9816217152544281 \\
916 & 0.9816217161486727 \\
917 & 0.9816217204069805 \\
918 & 0.9816217204069806 \\
919 & 0.98162172125021 \\
920 & 0.9816217212502102 \\
921 & 0.9816281626707074 \\
922 & 0.981634540314764 \\
923 & 0.9816408548138298 \\
924 & 0.9816408556241564 \\
925 & 0.98164085642646 \\
926 & 0.9816408572208203 \\
927 & 0.9816408580073156 \\
928 & 0.9816785825577404 \\
929 & 0.9816785840998877 \\
930 & 0.9817221863088585 \\
931 & 0.981728375585043 \\
932 & 0.9817283763484823 \\
933 & 0.981740512184138 \\
934 & 0.9817774941677656 \\
935 & 0.9817774949161622 \\
936 & 0.9818141114605837 \\
937 & 0.9818141128992544 \\
938 & 0.9818503661629304 \\
939 & 0.9818503661629306 \\
940 & 0.9818563740023641 \\
941 & 0.9818563740023644 \\
942 & 0.9818563747075258 \\
943 & 0.9818681547848463 \\
944 & 0.9818681554830263 \\
945 & 0.9818681554830265 \\
946 & 0.9818681561742936 \\
947 & 0.9818681568587165 \\
948 & 0.9818681582007224 \\
949 & 0.9818681595164142 \\
950 & 0.9818853156201556 \\
951 & 0.981890977485637 \\
952 & 0.981890978130521 \\
953 & 0.981896584576427 \\
954 & 0.9818965845764273 \\
955 & 0.9819021361327493 \\
956 & 0.9819021373601596 \\
957 & 0.9819021385635028 \\
958 & 0.9819380321788124 \\
959 & 0.981938032774527 \\
960 & 0.9819380327745272 \\
961 & 0.9819380333643434 \\
962 & 0.9819435292991752 \\
963 & 0.9819489719415185 \\
964 & 0.9819845112861993 \\
965 & 0.9819845118365466 \\
966 & 0.9819845118365466 \\
967 & 0.9819845123814447 \\
968 & 0.9819845123814448 \\
969 & 0.9819845134498724 \\
970 & 0.9819845144973505 \\
971 & 0.9819845150159041 \\
972 & 0.981984516526254 \\
973 & 0.9819899045841443 \\
974 & 0.9819952392853445 \\
975 & 0.9819952402341688 \\
976 & 0.9819952407038839 \\
977 & 0.9819952411689483 \\
978 & 0.9819952411689484 \\
979 & 0.9819952420808393 \\
980 & 0.9819952425322704 \\
981 & 0.9819952425322706 \\
982 & 0.982030429348519 \\
983 & 0.9820304297910553 \\
984 & 0.9820304310799954 \\
985 & 0.9820304315053883 \\
986 & 0.9820304315053884 \\
987 & 0.9820304315053884 \\
988 & 0.9820652699159624 \\
989 & 0.9820652699159624 \\
990 & 0.9820652703329734 \\
991 & 0.9820652707458556 \\
992 & 0.9820652711546497 \\
993 & 0.982065271956207 \\
994 & 0.9820652727420472 \\
995 & 0.982065273512479 \\
996 & 0.9820705544366369 \\
997 & 0.9820705548180387 \\
998 & 0.9820705548180387 \\
999 & 0.9820809095712894 \\
}\expectedgaincenti

\pgfplotstableread[row sep=\\,col sep=&]{
values & P \\
0 & 0.5525353589016282 \\
1 & 0.6303661708231101 \\
2 & 0.6733976302127822 \\
3 & 0.700551265445668 \\
4 & 0.7193527473728801 \\
5 & 0.7333401984043427 \\
6 & 0.7450626950111878 \\
7 & 0.7542905601793841 \\
8 & 0.761355571412957 \\
9 & 0.7678184935144412 \\
10 & 0.7728177320316557 \\
11 & 0.7773556125943928 \\
12 & 0.781404779293789 \\
13 & 0.7847268669783732 \\
14 & 0.7875649205089057 \\
15 & 0.7902448286385246 \\
16 & 0.7928970153601597 \\
17 & 0.7951930209001534 \\
18 & 0.7971987368691178 \\
19 & 0.7994997957705303 \\
20 & 0.8017001815301597 \\
21 & 0.8034331848002574 \\
22 & 0.8050980667816945 \\
23 & 0.8070905419892743 \\
24 & 0.8086462264162598 \\
25 & 0.8096349128749908 \\
26 & 0.8106866425543535 \\
27 & 0.8118488329987493 \\
28 & 0.8127420208421681 \\
29 & 0.8138696502833029 \\
30 & 0.815282859311553 \\
31 & 0.8160711693715088 \\
32 & 0.8172649168543369 \\
33 & 0.8182155657275396 \\
34 & 0.8193674383747225 \\
35 & 0.8203664562138607 \\
36 & 0.8211503822396339 \\
37 & 0.8217483905261278 \\
38 & 0.8225791251337828 \\
39 & 0.8234013741321017 \\
40 & 0.8241641213403221 \\
41 & 0.8251446415770314 \\
42 & 0.8258935805417025 \\
43 & 0.8265779121424245 \\
44 & 0.8272543663705381 \\
45 & 0.8278432336053024 \\
46 & 0.8283658905188441 \\
47 & 0.8292908221993263 \\
48 & 0.8300462888773519 \\
49 & 0.8306779540475859 \\
50 & 0.8314364154290099 \\
51 & 0.83184711032664 \\
52 & 0.832513876577025 \\
53 & 0.8331707869327346 \\
54 & 0.8337397518348467 \\
55 & 0.8343608420321497 \\
56 & 0.8345729680561367 \\
57 & 0.8351029335043818 \\
58 & 0.8354869045956551 \\
59 & 0.8360225037460033 \\
60 & 0.8366079450307389 \\
61 & 0.8374397847670015 \\
62 & 0.838183315027079 \\
63 & 0.8389193682815654 \\
64 & 0.8392872018483224 \\
65 & 0.839845593386068 \\
66 & 0.8402693068079006 \\
67 & 0.8407297752426408 \\
68 & 0.8413971966370899 \\
69 & 0.8417342547104733 \\
70 & 0.842169822820502 \\
71 & 0.8424633988835081 \\
72 & 0.842986181112338 \\
73 & 0.8435378572254534 \\
74 & 0.8438263179119309 \\
75 & 0.8440560778066112 \\
76 & 0.8444478640931247 \\
77 & 0.8448543111030067 \\
78 & 0.8451948947434282 \\
79 & 0.8456666185919275 \\
80 & 0.8461843897767908 \\
81 & 0.8465647385888184 \\
82 & 0.8470135696413481 \\
83 & 0.8474408926100111 \\
84 & 0.847878065762134 \\
85 & 0.8482506855199055 \\
86 & 0.848686102106634 \\
87 & 0.8491466716329366 \\
88 & 0.8495107752294625 \\
89 & 0.8497147085917265 \\
90 & 0.850075175274582 \\
91 & 0.8506924801888621 \\
92 & 0.8512662226320415 \\
93 & 0.8517776630213825 \\
94 & 0.8521269205418689 \\
95 & 0.8522665305159297 \\
96 & 0.8524729165509792 \\
97 & 0.8530307686850053 \\
98 & 0.8533947103783032 \\
99 & 0.8535788635513728 \\
100 & 0.8538512861890474 \\
101 & 0.8542544737876812 \\
102 & 0.8546216595146453 \\
103 & 0.8549097927728404 \\
104 & 0.855273336094997 \\
105 & 0.8556350691685426 \\
106 & 0.8560802492316846 \\
107 & 0.8563631514292114 \\
108 & 0.8567196736276999 \\
109 & 0.857077686361281 \\
110 & 0.8574442028127592 \\
111 & 0.8575956300712175 \\
112 & 0.8578517205322933 \\
113 & 0.8580237525390385 \\
114 & 0.8583110076559811 \\
115 & 0.8587021837185476 \\
116 & 0.8590385781709411 \\
117 & 0.8593952752439776 \\
118 & 0.8597505041915491 \\
119 & 0.8600808147401048 \\
120 & 0.860348910530066 \\
121 & 0.860632542097715 \\
122 & 0.8609336479713297 \\
123 & 0.8612762210484132 \\
124 & 0.8616648572972224 \\
125 & 0.862075045134907 \\
126 & 0.8624181212591 \\
127 & 0.862747848950974 \\
128 & 0.8630754806214371 \\
129 & 0.8633748342663784 \\
130 & 0.8637034827019479 \\
131 & 0.8639084819694026 \\
132 & 0.8641774685977789 \\
133 & 0.8644219012109814 \\
134 & 0.8646356862506283 \\
135 & 0.8649123639236621 \\
136 & 0.8652257696169302 \\
137 & 0.8654668538463544 \\
138 & 0.8658063615759622 \\
139 & 0.8660432511360764 \\
140 & 0.866193340696756 \\
141 & 0.8664666229786381 \\
142 & 0.8666807969174882 \\
143 & 0.8671116292307691 \\
144 & 0.8673256357113607 \\
145 & 0.8675486423509556 \\
146 & 0.8677079692393703 \\
147 & 0.8680241961689285 \\
148 & 0.8683160365291591 \\
149 & 0.8685634839333175 \\
150 & 0.868784962999815 \\
151 & 0.8690157484050444 \\
152 & 0.8692270991909546 \\
153 & 0.8694270944147406 \\
154 & 0.8696108943963287 \\
155 & 0.8698426415476791 \\
156 & 0.8701440876706764 \\
157 & 0.8704177085123596 \\
158 & 0.8706446470776878 \\
159 & 0.8708979237365166 \\
160 & 0.8710178689605661 \\
161 & 0.8712646330137673 \\
162 & 0.8716104260900566 \\
163 & 0.871928725480229 \\
164 & 0.8720581916198887 \\
165 & 0.8722207773422376 \\
166 & 0.8723832184226097 \\
167 & 0.8726036151336041 \\
168 & 0.8728086041726827 \\
169 & 0.8729935867652178 \\
170 & 0.8731812336907128 \\
171 & 0.8733504076547901 \\
172 & 0.8734934534809222 \\
173 & 0.8737352645127047 \\
174 & 0.873976835769157 \\
175 & 0.8741199768489036 \\
176 & 0.874319882802194 \\
177 & 0.8746017616390079 \\
178 & 0.8747251417321694 \\
179 & 0.8747936073468473 \\
180 & 0.8749711258794349 \\
181 & 0.8751636968539982 \\
182 & 0.8754719204187419 \\
183 & 0.875566178689026 \\
184 & 0.8756989813887293 \\
185 & 0.8760263008263838 \\
186 & 0.8763478811329879 \\
187 & 0.8765730456594653 \\
188 & 0.8768861482621625 \\
189 & 0.8770566338890362 \\
190 & 0.8771854912839773 \\
191 & 0.8773998320134824 \\
192 & 0.8776019056509932 \\
193 & 0.877799301665829 \\
194 & 0.8779929285727084 \\
195 & 0.8781041487085807 \\
196 & 0.8782602081667792 \\
197 & 0.8783968153096317 \\
198 & 0.878511435976928 \\
199 & 0.8787337121452102 \\
200 & 0.8788301170038569 \\
201 & 0.8790681891325925 \\
202 & 0.8792453502265649 \\
203 & 0.8793495327371084 \\
204 & 0.8794940647271265 \\
205 & 0.8796067641843458 \\
206 & 0.8798250496635522 \\
207 & 0.8800169714650489 \\
208 & 0.8801479620555301 \\
209 & 0.8802189494787863 \\
210 & 0.8803234161648844 \\
211 & 0.8804050299384726 \\
212 & 0.8804783704081284 \\
213 & 0.8806898445084411 \\
214 & 0.8808970143825641 \\
215 & 0.8810311050795823 \\
216 & 0.8810959437998317 \\
217 & 0.8811900083796634 \\
218 & 0.8814449594370538 \\
219 & 0.8816059532138244 \\
220 & 0.8817485185043257 \\
221 & 0.881830554916461 \\
222 & 0.881922952695282 \\
223 & 0.8821851871603741 \\
224 & 0.882280817435024 \\
225 & 0.8823624575843331 \\
226 & 0.8824294732110839 \\
227 & 0.8825420015265717 \\
228 & 0.882745537163924 \\
229 & 0.8828501423255258 \\
230 & 0.8831541408334327 \\
231 & 0.8833517379191917 \\
232 & 0.8834312314064543 \\
233 & 0.8835808128495783 \\
234 & 0.8836856501746505 \\
235 & 0.8837924224286017 \\
236 & 0.8838521967568488 \\
237 & 0.8841167171911466 \\
238 & 0.8842718385111904 \\
239 & 0.8843544456139659 \\
240 & 0.8844650463316385 \\
241 & 0.8846271546996042 \\
242 & 0.884768187717369 \\
243 & 0.8849534446215986 \\
244 & 0.8851704430124345 \\
245 & 0.8853729281431788 \\
246 & 0.8855628568641521 \\
247 & 0.8857377345321736 \\
248 & 0.8859428014773497 \\
249 & 0.8860412371223546 \\
250 & 0.8862388075425637 \\
251 & 0.8864237191723457 \\
252 & 0.8865436147896414 \\
253 & 0.8866707031796678 \\
254 & 0.8867770144532813 \\
255 & 0.8868921412634247 \\
256 & 0.8869609079182613 \\
257 & 0.8870528087745531 \\
258 & 0.8871176036668609 \\
259 & 0.887231860634185 \\
260 & 0.8873323103784815 \\
261 & 0.887459558769756 \\
262 & 0.8876002405320434 \\
263 & 0.8877754010310052 \\
264 & 0.8878518394537762 \\
265 & 0.8879954125740125 \\
266 & 0.8881640596714286 \\
267 & 0.888253175052552 \\
268 & 0.8883333606728597 \\
269 & 0.888439029979157 \\
270 & 0.8885397817890728 \\
271 & 0.888662804050498 \\
272 & 0.8887149177331791 \\
273 & 0.8888164346951499 \\
274 & 0.8889343341230869 \\
275 & 0.8890127101142588 \\
276 & 0.8891423239035648 \\
277 & 0.8891977995585443 \\
278 & 0.8894257452046443 \\
279 & 0.8895111241357516 \\
280 & 0.8895925158239539 \\
281 & 0.8897223587226853 \\
282 & 0.8899163956362424 \\
283 & 0.8899485591036368 \\
284 & 0.8900606185415469 \\
285 & 0.8902337891550621 \\
286 & 0.8904045413968654 \\
287 & 0.890496477083753 \\
288 & 0.8906303368057463 \\
289 & 0.8907408100620642 \\
290 & 0.8909031864286525 \\
291 & 0.8909700475094621 \\
292 & 0.8911293518865079 \\
293 & 0.8911934026607283 \\
294 & 0.8913226122481067 \\
295 & 0.8914446238974665 \\
296 & 0.8916810968368148 \\
297 & 0.8917908428451196 \\
298 & 0.891998198329308 \\
299 & 0.892133773923239 \\
300 & 0.8922707533353568 \\
301 & 0.8923973280381895 \\
302 & 0.8925545932361694 \\
303 & 0.8926609780500906 \\
304 & 0.892729479092007 \\
305 & 0.8927764987874607 \\
306 & 0.8928827383270044 \\
307 & 0.8929603470271132 \\
308 & 0.8930915460284062 \\
309 & 0.8931399679484284 \\
310 & 0.8931818585300616 \\
311 & 0.8932236660471256 \\
312 & 0.8933679683096977 \\
313 & 0.8934373323028005 \\
314 & 0.89349900792649 \\
315 & 0.8936090026055964 \\
316 & 0.8937946027421824 \\
317 & 0.893828853917765 \\
318 & 0.8939518998054597 \\
319 & 0.8940868163088053 \\
320 & 0.8941470994107088 \\
321 & 0.8942185219138454 \\
322 & 0.8942940903458841 \\
323 & 0.8944854962333737 \\
324 & 0.8945560463025759 \\
325 & 0.8945894921285681 \\
326 & 0.8946752646846071 \\
327 & 0.8947784074960904 \\
328 & 0.8948367865495157 \\
329 & 0.8949889249581039 \\
330 & 0.8950794773035237 \\
331 & 0.8952171856615653 \\
332 & 0.8954495489335355 \\
333 & 0.8955194862530439 \\
334 & 0.8956013613154784 \\
335 & 0.8956718682672249 \\
336 & 0.8957472922420143 \\
337 & 0.8958091571140103 \\
338 & 0.8958663957573798 \\
339 & 0.8959872331572906 \\
340 & 0.8960283508083737 \\
341 & 0.8961233466415907 \\
342 & 0.8962398120121482 \\
343 & 0.896350465890129 \\
344 & 0.8964633755002653 \\
345 & 0.8964999340139463 \\
346 & 0.8966424967476588 \\
347 & 0.8967692495449966 \\
348 & 0.8968842270208244 \\
349 & 0.897015461688734 \\
350 & 0.8970932927526584 \\
351 & 0.8972265775432101 \\
352 & 0.8973627614611556 \\
353 & 0.8975094645215186 \\
354 & 0.8975952981837577 \\
355 & 0.8976669646785286 \\
356 & 0.8977136629733091 \\
357 & 0.8977956802658679 \\
358 & 0.8979104935136931 \\
359 & 0.8980088458287767 \\
360 & 0.8981092871042848 \\
361 & 0.8982253305222134 \\
362 & 0.8982709163063979 \\
363 & 0.8983875633016236 \\
364 & 0.8983875633018266 \\
365 & 0.8984919699571698 \\
366 & 0.8985424652370005 \\
367 & 0.898571872918202 \\
368 & 0.8986359786001125 \\
369 & 0.8986996538467734 \\
370 & 0.8987391299541206 \\
371 & 0.8987973679556123 \\
372 & 0.8988935164291647 \\
373 & 0.8989569936769303 \\
374 & 0.8990820752144725 \\
375 & 0.8991452429471344 \\
376 & 0.8992442603591811 \\
377 & 0.8993066360678679 \\
378 & 0.8994712758440556 \\
379 & 0.8994811522878272 \\
380 & 0.8995450411497277 \\
381 & 0.8995970059970819 \\
382 & 0.8997109159467428 \\
383 & 0.8997448009732119 \\
384 & 0.8998059268774757 \\
385 & 0.8999190385640459 \\
386 & 0.8999854744062865 \\
387 & 0.9000780476258966 \\
388 & 0.900223571387568 \\
389 & 0.9002846927946536 \\
390 & 0.9003263774062582 \\
391 & 0.9004340730547143 \\
392 & 0.9005770754077772 \\
393 & 0.9006660059660606 \\
394 & 0.9007365996671566 \\
395 & 0.9007975974959207 \\
396 & 0.9008711632628069 \\
397 & 0.9009318584746457 \\
398 & 0.9010736807836619 \\
399 & 0.901185074853048 \\
400 & 0.9012764691547999 \\
401 & 0.9014790313152037 \\
402 & 0.9015881590397578 \\
403 & 0.9017032588251646 \\
404 & 0.9017892304221146 \\
405 & 0.9018556611541211 \\
406 & 0.9019088792261717 \\
407 & 0.9019985068471164 \\
408 & 0.9021263670345319 \\
409 & 0.902174493756583 \\
410 & 0.9022505027946623 \\
411 & 0.9022766272096426 \\
412 & 0.90233308350803 \\
413 & 0.9024685856124726 \\
414 & 0.9025396396040748 \\
415 & 0.902661694791946 \\
416 & 0.9027558167822032 \\
417 & 0.9027984630877718 \\
418 & 0.902851285731928 \\
419 & 0.9029127951629224 \\
420 & 0.9030389350011306 \\
421 & 0.9031314346921832 \\
422 & 0.9031752198569986 \\
423 & 0.9033027078458044 \\
424 & 0.9033270470362917 \\
425 & 0.9033391802319584 \\
426 & 0.9034447544772749 \\
427 & 0.903513030350266 \\
428 & 0.9035601630959611 \\
429 & 0.9036059490835587 \\
430 & 0.9036296415956004 \\
431 & 0.9036947819491695 \\
432 & 0.90379048318417 \\
433 & 0.9038529129128214 \\
434 & 0.9039135602525079 \\
435 & 0.9039856582792273 \\
436 & 0.9039971553985946 \\
437 & 0.9040728543463823 \\
438 & 0.9041327935677674 \\
439 & 0.9042557185064906 \\
440 & 0.9044016469808417 \\
441 & 0.9044599308588404 \\
442 & 0.9044821808097687 \\
443 & 0.9045135047832839 \\
444 & 0.9045585755555537 \\
445 & 0.9046370909419651 \\
446 & 0.9047470629991189 \\
447 & 0.9048317686168591 \\
448 & 0.9048532556111147 \\
449 & 0.9049404082367212 \\
450 & 0.9050243995199793 \\
451 & 0.905113974750389 \\
452 & 0.9051418912112427 \\
453 & 0.9052019144504913 \\
454 & 0.9052520846044851 \\
455 & 0.9052788480650049 \\
456 & 0.9053254675186038 \\
457 & 0.9054107001953456 \\
458 & 0.9054408366987037 \\
459 & 0.905467529354041 \\
460 & 0.9055140213446982 \\
461 & 0.9055927358935414 \\
462 & 0.9056835785655731 \\
463 & 0.905729766031733 \\
464 & 0.9057983752779142 \\
465 & 0.9058443254843583 \\
466 & 0.9059029624693504 \\
467 & 0.9059777130389474 \\
468 & 0.9060356831078245 \\
469 & 0.9060778836116514 \\
470 & 0.9061872048298539 \\
471 & 0.9062645844867566 \\
472 & 0.9063449927056811 \\
473 & 0.9063929538497434 \\
474 & 0.9064249015583739 \\
475 & 0.9065271565663255 \\
476 & 0.9065793915076427 \\
477 & 0.9066559064943908 \\
478 & 0.9066874903410042 \\
479 & 0.906734554637407 \\
480 & 0.9068149848447764 \\
481 & 0.9068523743144431 \\
482 & 0.906912241796642 \\
483 & 0.9069433002512913 \\
484 & 0.9069810389585821 \\
485 & 0.9070627158257052 \\
486 & 0.9071088869355932 \\
487 & 0.9071676687911278 \\
488 & 0.9072139552198701 \\
489 & 0.9072326395886607 \\
490 & 0.9072855826181143 \\
491 & 0.9073002654431335 \\
492 & 0.9073689908821154 \\
493 & 0.9074438044126135 \\
494 & 0.9074746385862206 \\
495 & 0.9074920135257156 \\
496 & 0.9075510390935002 \\
497 & 0.9076303734910599 \\
498 & 0.9076822253969403 \\
499 & 0.9077034244008542 \\
500 & 0.9077368185979863 \\
501 & 0.9077846904957908 \\
502 & 0.9078575469025895 \\
503 & 0.9079238431409145 \\
504 & 0.9079629310811966 \\
505 & 0.9080211264844966 \\
506 & 0.9081057482289187 \\
507 & 0.9081676955132127 \\
508 & 0.9082847012333168 \\
509 & 0.9083162400869551 \\
510 & 0.9083929706792071 \\
511 & 0.9084340304576105 \\
512 & 0.9085282693171941 \\
513 & 0.9086180859490948 \\
514 & 0.9086588381621693 \\
515 & 0.9086898726859592 \\
516 & 0.9087522159043343 \\
517 & 0.9088293071263852 \\
518 & 0.9088398851912247 \\
519 & 0.9088896728803239 \\
520 & 0.9089124152494978 \\
521 & 0.9089916720889372 \\
522 & 0.9090385855375749 \\
523 & 0.9091123462404563 \\
524 & 0.9091447612503107 \\
525 & 0.9091793025398178 \\
526 & 0.9091966303008522 \\
527 & 0.9092724915957863 \\
528 & 0.9093001107188222 \\
529 & 0.9093177921474165 \\
530 & 0.9094139676392224 \\
531 & 0.909494211200746 \\
532 & 0.9095042045472039 \\
533 & 0.9095610763379742 \\
534 & 0.9095832973449733 \\
535 & 0.9096729083631121 \\
536 & 0.9097097674732 \\
537 & 0.9097794579363494 \\
538 & 0.9097963301073498 \\
539 & 0.9098595946133983 \\
540 & 0.9099388246466806 \\
541 & 0.9099459553404339 \\
542 & 0.9100145779673307 \\
543 & 0.9100509791749297 \\
544 & 0.9101085721673174 \\
545 & 0.9101226356930537 \\
546 & 0.9101512245003528 \\
547 & 0.9101866014951424 \\
548 & 0.9101958013313108 \\
549 & 0.9102026805707191 \\
550 & 0.9102386701021711 \\
551 & 0.9103025172215484 \\
552 & 0.9103450791000194 \\
553 & 0.91039209386913 \\
554 & 0.9104345661818887 \\
555 & 0.9104579771923949 \\
556 & 0.91051922112996 \\
557 & 0.9105658241894069 \\
558 & 0.9106488444153665 \\
559 & 0.9107009191312401 \\
560 & 0.910707529263952 \\
561 & 0.910730685120271 \\
562 & 0.9107516123220039 \\
563 & 0.9108124063573536 \\
564 & 0.9108618552571911 \\
565 & 0.9109275244468823 \\
566 & 0.9109760043406923 \\
567 & 0.9110270256957834 \\
568 & 0.9110544042068593 \\
569 & 0.9110871048122082 \\
570 & 0.9111119075599533 \\
571 & 0.9111879937331666 \\
572 & 0.9112772657176472 \\
573 & 0.9113253201457597 \\
574 & 0.9114222751077178 \\
575 & 0.9114384498148109 \\
576 & 0.9114769359540481 \\
577 & 0.9115361234561191 \\
578 & 0.9115610253299128 \\
579 & 0.9116231012793544 \\
580 & 0.9116707283690411 \\
581 & 0.9116968894003109 \\
582 & 0.9117090106802066 \\
583 & 0.9117477221386511 \\
584 & 0.9118378593398807 \\
585 & 0.9118378593398807 \\
586 & 0.9118844085015182 \\
587 & 0.9119082173697035 \\
588 & 0.912000933746005 \\
589 & 0.9120531498320767 \\
590 & 0.9120905444984462 \\
591 & 0.9121317061610213 \\
592 & 0.9121650150554582 \\
593 & 0.9122078422761964 \\
594 & 0.9122847444898322 \\
595 & 0.9123197645844289 \\
596 & 0.912342927704465 \\
597 & 0.9124120385020761 \\
598 & 0.9124411948950015 \\
599 & 0.912482877895479 \\
600 & 0.9125263547402133 \\
601 & 0.912551076097494 \\
602 & 0.9126178088190959 \\
603 & 0.9126419981207697 \\
604 & 0.9127187642005656 \\
605 & 0.912760058823466 \\
606 & 0.9127943269065216 \\
607 & 0.9128437396772721 \\
608 & 0.9129277732290273 \\
609 & 0.9129724830801391 \\
610 & 0.9130844806734149 \\
611 & 0.9131165283316428 \\
612 & 0.9131352058994594 \\
613 & 0.9132166338758352 \\
614 & 0.9133251657052172 \\
615 & 0.9133621903609203 \\
616 & 0.913367466367767 \\
617 & 0.9133928996555649 \\
618 & 0.9134561467176402 \\
619 & 0.9134596256755766 \\
620 & 0.9135216610121599 \\
621 & 0.9136096130783218 \\
622 & 0.9136399979297731 \\
623 & 0.9136816989156491 \\
624 & 0.913727021659741 \\
625 & 0.9137434486052592 \\
626 & 0.9137484781013431 \\
627 & 0.9137828945178883 \\
628 & 0.9138193628843608 \\
629 & 0.9138423040544378 \\
630 & 0.9138836299876696 \\
631 & 0.9139145008735924 \\
632 & 0.9139453052402925 \\
633 & 0.9139848342329172 \\
634 & 0.9140412954577535 \\
635 & 0.9140590750585472 \\
636 & 0.9141079077068067 \\
637 & 0.9141359878364034 \\
638 & 0.9141984557589299 \\
639 & 0.9142000538553245 \\
640 & 0.914331049360282 \\
641 & 0.914422572283 \\
642 & 0.9144956338332999 \\
643 & 0.9145294952163355 \\
644 & 0.9145682450254627 \\
645 & 0.9145973117729264 \\
646 & 0.9146470854917301 \\
647 & 0.9147281671531083 \\
648 & 0.9147945556298541 \\
649 & 0.9148537679167102 \\
650 & 0.9148987561996264 \\
651 & 0.9149370719829091 \\
652 & 0.9149642275187708 \\
653 & 0.9150230890903334 \\
654 & 0.9150621598462455 \\
655 & 0.9151281646023802 \\
656 & 0.9151340786387072 \\
657 & 0.9151872933245442 \\
658 & 0.9152042009534994 \\
659 & 0.915229666033115 \\
660 & 0.9152634035472675 \\
661 & 0.9152802548148303 \\
662 & 0.9152970805349211 \\
663 & 0.9153109850998404 \\
664 & 0.9153706032256151 \\
665 & 0.915418670730509 \\
666 & 0.9154511181711353 \\
667 & 0.9154706455375223 \\
668 & 0.9155124804502568 \\
669 & 0.9155181179973965 \\
670 & 0.9155870017501174 \\
671 & 0.9156269689884816 \\
672 & 0.915657106478406 \\
673 & 0.9157247558063601 \\
674 & 0.915766095291025 \\
675 & 0.9158321454944208 \\
676 & 0.915876910015994 \\
677 & 0.9159152800245491 \\
678 & 0.9159395948641884 \\
679 & 0.9160079965732887 \\
680 & 0.916089663242712 \\
681 & 0.9160950213285026 \\
682 & 0.9161678592752985 \\
683 & 0.9161825589502487 \\
684 & 0.9162435828345968 \\
685 & 0.9162595542929759 \\
686 & 0.9162648117484357 \\
687 & 0.9163397375287794 \\
688 & 0.9164013550480582 \\
689 & 0.9164674768479736 \\
690 & 0.916507076657201 \\
691 & 0.9165215543650328 \\
692 & 0.9165806860315424 \\
693 & 0.9166082395415236 \\
694 & 0.9166239121123104 \\
695 & 0.9166420923201497 \\
696 & 0.9166484070211681 \\
697 & 0.9166758051346957 \\
698 & 0.916696177662084 \\
699 & 0.9167079761762603 \\
700 & 0.916711738603411 \\
701 & 0.9167698749775998 \\
702 & 0.9168186517700159 \\
703 & 0.9168198946974466 \\
704 & 0.9169157843974768 \\
705 & 0.9169311838455323 \\
706 & 0.9169594453673692 \\
707 & 0.9169606747189851 \\
708 & 0.9170256072146297 \\
709 & 0.9170457619679027 \\
710 & 0.9171001213226676 \\
711 & 0.917145871676188 \\
712 & 0.91717317932951 \\
713 & 0.9172291573784036 \\
714 & 0.9172582011267985 \\
715 & 0.9172848042599809 \\
716 & 0.9172930649333373 \\
717 & 0.9173069279385287 \\
718 & 0.917340814577808 \\
719 & 0.917365505566952 \\
720 & 0.9173931066152723 \\
721 & 0.9174103578933979 \\
722 & 0.9174355466191262 \\
723 & 0.9174629825057369 \\
724 & 0.917525232835815 \\
725 & 0.9175827003062542 \\
726 & 0.9175997121021897 \\
727 & 0.9176252476309149 \\
728 & 0.9176399566625623 \\
729 & 0.9177045810013781 \\
730 & 0.9177255449411753 \\
731 & 0.9177995425054352 \\
732 & 0.9178377929712238 \\
733 & 0.9178534548258833 \\
734 & 0.9178926237871556 \\
735 & 0.9179439023550532 \\
736 & 0.9180313007755766 \\
737 & 0.918118452642476 \\
738 & 0.9181778163213853 \\
739 & 0.9182117521871108 \\
740 & 0.9182752374562865 \\
741 & 0.9183200744353296 \\
742 & 0.9183528769843222 \\
743 & 0.9183570648374366 \\
744 & 0.918399835671761 \\
745 & 0.9184149892859371 \\
746 & 0.9184160289569548 \\
747 & 0.9184551034713178 \\
748 & 0.918485177667088 \\
749 & 0.9185277114088423 \\
750 & 0.9185368382616476 \\
751 & 0.9185626633047135 \\
752 & 0.9185696845378475 \\
753 & 0.9185855562842207 \\
754 & 0.9186190434754838 \\
755 & 0.9186623363228573 \\
756 & 0.918666284902704 \\
757 & 0.9186810558753735 \\
758 & 0.9186948239861914 \\
759 & 0.9187447660422119 \\
760 & 0.9188131111002521 \\
761 & 0.9188483514892177 \\
762 & 0.9188805360475046 \\
763 & 0.9189136464254842 \\
764 & 0.9189652445776281 \\
765 & 0.9189885039995315 \\
766 & 0.9190340680615995 \\
767 & 0.9190447556586577 \\
768 & 0.9191047941389242 \\
769 & 0.9191540784416651 \\
770 & 0.9191791373732319 \\
771 & 0.9192301889226505 \\
772 & 0.9192762625680475 \\
773 & 0.919279996540287 \\
774 & 0.91938130613308 \\
775 & 0.9194145328288711 \\
776 & 0.9194470650967973 \\
777 & 0.919448911692687 \\
778 & 0.9194869157469246 \\
779 & 0.9195124925712569 \\
780 & 0.9195239010976036 \\
781 & 0.919564017147081 \\
782 & 0.919600782040639 \\
783 & 0.9196338293352444 \\
784 & 0.919687253578823 \\
785 & 0.919697636002736 \\
786 & 0.9197742650666293 \\
787 & 0.9198509038752737 \\
788 & 0.9199055461129768 \\
789 & 0.9199202569745601 \\
790 & 0.9199591778068595 \\
791 & 0.9199703542489247 \\
792 & 0.9200201630016136 \\
793 & 0.920067163078292 \\
794 & 0.9200680318893792 \\
795 & 0.9200799839047321 \\
796 & 0.9200834383945067 \\
797 & 0.9201151946773403 \\
798 & 0.9201596384643144 \\
799 & 0.9201613434589264 \\
800 & 0.9201630450503557 \\
801 & 0.9202023006274717 \\
802 & 0.9202259801697521 \\
803 & 0.920270616817391 \\
804 & 0.9203004479682424 \\
805 & 0.9203403135546111 \\
806 & 0.9203444876266603 \\
807 & 0.9203658312140942 \\
808 & 0.9203683182184492 \\
809 & 0.9204193952797063 \\
810 & 0.9204382009850433 \\
811 & 0.9204732553452256 \\
812 & 0.9205200226741231 \\
813 & 0.920552886676346 \\
814 & 0.9205837482143451 \\
815 & 0.92059861030707 \\
816 & 0.9206219095376652 \\
817 & 0.9206351235214626 \\
818 & 0.9206870876894865 \\
819 & 0.9207364485435154 \\
820 & 0.9207567171602024 \\
821 & 0.9207888828505023 \\
822 & 0.9208182437208795 \\
823 & 0.9208590390807107 \\
824 & 0.9209013308337157 \\
825 & 0.920947553288214 \\
826 & 0.9209829548664108 \\
827 & 0.9210064987248051 \\
828 & 0.9210103487926461 \\
829 & 0.9210685155261144 \\
830 & 0.9210888777832394 \\
831 & 0.9211088835732647 \\
832 & 0.9211209566331707 \\
833 & 0.9211224767991033 \\
834 & 0.9211450424442589 \\
835 & 0.9211578175527597 \\
836 & 0.9212066757307958 \\
837 & 0.9212380799942491 \\
838 & 0.9212587213529125 \\
839 & 0.9213098814470018 \\
840 & 0.9213297224968081 \\
841 & 0.9213319310826038 \\
842 & 0.921370352021488 \\
843 & 0.9213725496061748 \\
844 & 0.9213754680453339 \\
845 & 0.9213969549757935 \\
846 & 0.9214388683272597 \\
847 & 0.9214499336197841 \\
848 & 0.921480707218405 \\
849 & 0.9214982627292116 \\
850 & 0.9215653459035885 \\
851 & 0.9215931665975837 \\
852 & 0.9216443102447978 \\
853 & 0.9216552746874174 \\
854 & 0.9216916254969545 \\
855 & 0.9217111918864073 \\
856 & 0.9217395700447565 \\
857 & 0.9217614108119871 \\
858 & 0.9217818086564349 \\
859 & 0.9218471105734468 \\
860 & 0.9218498897427857 \\
861 & 0.9218708717685334 \\
862 & 0.9219178012967882 \\
863 & 0.9219413017579211 \\
864 & 0.9220011640529617 \\
865 & 0.9220413459971014 \\
866 & 0.9220700686618959 \\
867 & 0.9220807926475756 \\
868 & 0.9220987599408453 \\
869 & 0.9221108084257913 \\
870 & 0.9221294233724846 \\
871 & 0.9221521104070409 \\
872 & 0.9221906793930595 \\
873 & 0.9222112830923058 \\
874 & 0.9222219031685952 \\
875 & 0.9222325113213395 \\
876 & 0.9222351316245888 \\
877 & 0.9222808234881489 \\
878 & 0.9223315978269169 \\
879 & 0.9223593285437434 \\
880 & 0.9223870302668514 \\
881 & 0.9223889624629416 \\
882 & 0.9224026491780858 \\
883 & 0.9224064638197101 \\
884 & 0.9224156642167376 \\
885 & 0.9224531036458452 \\
886 & 0.9224813290247557 \\
887 & 0.9224930105540688 \\
888 & 0.9224961421870389 \\
889 & 0.9224998777333254 \\
890 & 0.9225133601176256 \\
891 & 0.9225438175351168 \\
892 & 0.9226118812837044 \\
893 & 0.9226302083340534 \\
894 & 0.922631427704964 \\
895 & 0.9226429656640882 \\
896 & 0.9226709699705328 \\
897 & 0.9226983402581173 \\
898 & 0.9227104280468774 \\
899 & 0.9227110286717708 \\
900 & 0.9227280826992063 \\
901 & 0.9227928684229797 \\
902 & 0.9227970186697155 \\
903 & 0.9228236594065059 \\
904 & 0.922919731921113 \\
905 & 0.9229696874560515 \\
906 & 0.923030066178209 \\
907 & 0.9230487839613054 \\
908 & 0.9230505274843842 \\
909 & 0.9230785007411441 \\
910 & 0.9231086931001607 \\
911 & 0.9231463000568197 \\
912 & 0.9231784758973498 \\
913 & 0.9232048579022567 \\
914 & 0.923258277055778 \\
915 & 0.9233033250237707 \\
916 & 0.9233132647174243 \\
917 & 0.9233452182846414 \\
918 & 0.923380074817422 \\
919 & 0.9234003862638082 \\
920 & 0.9234026089308852 \\
921 & 0.9234472428294046 \\
922 & 0.9234662649238317 \\
923 & 0.9235155589726205 \\
924 & 0.9235328909946195 \\
925 & 0.9235692951220399 \\
926 & 0.9236131130790897 \\
927 & 0.9236239917603812 \\
928 & 0.9236705999300369 \\
929 & 0.9236809192906362 \\
930 & 0.9237196909350641 \\
931 & 0.9237542779759149 \\
932 & 0.9238082166371856 \\
933 & 0.9238517258221367 \\
934 & 0.9238538551048339 \\
935 & 0.9238651341652337 \\
936 & 0.9239459254373152 \\
937 & 0.923988718209961 \\
938 & 0.9240164869087819 \\
939 & 0.9240516108668908 \\
940 & 0.9240617608310108 \\
941 & 0.9240633184392344 \\
942 & 0.9240729338760189 \\
943 & 0.9240990395067017 \\
944 & 0.9241266586235132 \\
945 & 0.924137772038311 \\
946 & 0.9241473388662742 \\
947 & 0.9241568951180705 \\
948 & 0.9241749515937502 \\
949 & 0.9241939838288673 \\
950 & 0.9242453735536177 \\
951 & 0.9242643229711263 \\
952 & 0.9242842372628649 \\
953 & 0.924310609974904 \\
954 & 0.9243300069020642 \\
955 & 0.9243736703380843 \\
956 & 0.924383081446792 \\
957 & 0.9244098306369672 \\
958 & 0.924420194166443 \\
959 & 0.9244398930923202 \\
960 & 0.9244567313151235 \\
961 & 0.9244665665852276 \\
962 & 0.9244941639857566 \\
963 & 0.924512399969134 \\
964 & 0.9245380274136331 \\
965 & 0.9245413496947356 \\
966 & 0.9245674162006005 \\
967 & 0.9245855163556896 \\
968 & 0.9245961975817975 \\
969 & 0.9246147027060961 \\
970 & 0.9246322376419329 \\
971 & 0.9246442335725529 \\
972 & 0.9246708811242096 \\
973 & 0.924680977576342 \\
974 & 0.9247151087418046 \\
975 & 0.9247500950915849 \\
976 & 0.9247833253159415 \\
977 & 0.9248118096445063 \\
978 & 0.9248375590751625 \\
979 & 0.9248466259717144 \\
980 & 0.9248642742975229 \\
981 & 0.9249025087365152 \\
982 & 0.9249047283747616 \\
983 & 0.9249223132392222 \\
984 & 0.9249546978306795 \\
985 & 0.9249551399883715 \\
986 & 0.9249560225386549 \\
987 & 0.9249726700100339 \\
988 & 0.9250139865556243 \\
989 & 0.9250246905179459 \\
990 & 0.9250345047781181 \\
991 & 0.9250972323968673 \\
992 & 0.9251146094956129 \\
993 & 0.9251396465524039 \\
994 & 0.9251659437958889 \\
995 & 0.9251913539718719 \\
996 & 0.9252243017327817 \\
997 & 0.925257199775188 \\
998 & 0.9252752574079655 \\
999 & 0.9253321528348384 \\
}\expectedgainmili

\pgfplotstableread[row sep=\\,col sep=&]{
values & P \\
0 & 0.321779870985538 \\
1 & 0.3682579874074543 \\
2 & 0.40476008070689357 \\
3 & 0.4347112186811146 \\
4 & 0.4595766056499592 \\
5 & 0.48121953269770834 \\
6 & 0.500370667432365 \\
7 & 0.5169828618890291 \\
8 & 0.5317776052006948 \\
9 & 0.5449640599692005 \\
10 & 0.5567459166062645 \\
11 & 0.5674441612270988 \\
12 & 0.5770997945808702 \\
13 & 0.585869652534473 \\
14 & 0.5937529889452866 \\
15 & 0.6010748289809272 \\
16 & 0.6081089777007506 \\
17 & 0.6142768347192422 \\
18 & 0.6200961151229315 \\
19 & 0.6256081559300106 \\
20 & 0.6308337288484878 \\
21 & 0.6355483748873685 \\
22 & 0.6400173260676874 \\
23 & 0.6442340199361113 \\
24 & 0.6481891351502951 \\
25 & 0.6518971376323834 \\
26 & 0.6555337978376142 \\
27 & 0.6589307790106955 \\
28 & 0.6621626540831138 \\
29 & 0.6654760374830611 \\
30 & 0.6685101393577076 \\
31 & 0.6712082824336638 \\
32 & 0.6737886101662547 \\
33 & 0.6763577530918088 \\
34 & 0.6788938914288662 \\
35 & 0.6813162138491595 \\
36 & 0.6835885345970074 \\
37 & 0.6857486744355538 \\
38 & 0.6877737803424927 \\
39 & 0.6897715810885605 \\
40 & 0.6916709183293519 \\
41 & 0.6935206301298041 \\
42 & 0.695280601330618 \\
43 & 0.6969757889594983 \\
44 & 0.6986374375879498 \\
45 & 0.7001828379701659 \\
46 & 0.7017193462948019 \\
47 & 0.703252790064599 \\
48 & 0.7046750760049965 \\
49 & 0.7060167698081424 \\
50 & 0.7074672314595969 \\
51 & 0.7087623270620357 \\
52 & 0.7100332806817122 \\
53 & 0.7113146025317255 \\
54 & 0.7125215744559413 \\
55 & 0.7136440061502074 \\
56 & 0.7148369602175586 \\
57 & 0.7159882312816266 \\
58 & 0.7170008978599165 \\
59 & 0.71808242688287 \\
60 & 0.7190612867489139 \\
61 & 0.7200142337671478 \\
62 & 0.7209555340599951 \\
63 & 0.7218396069461779 \\
64 & 0.722791394276443 \\
65 & 0.7237611140292465 \\
66 & 0.7246775347468359 \\
67 & 0.7255221377071334 \\
68 & 0.7264090085623026 \\
69 & 0.7272318243473332 \\
70 & 0.7280763945459429 \\
71 & 0.7288101636722905 \\
72 & 0.7295710868269162 \\
73 & 0.7303464836180832 \\
74 & 0.7310893650417958 \\
75 & 0.7317877368593178 \\
76 & 0.7325069693865889 \\
77 & 0.7332085836589892 \\
78 & 0.7338908270286247 \\
79 & 0.7345513155736214 \\
80 & 0.7352015852513154 \\
81 & 0.7358515105193848 \\
82 & 0.7364853521047823 \\
83 & 0.7370993024339578 \\
84 & 0.7377083439441072 \\
85 & 0.7382815282775341 \\
86 & 0.7388013530092726 \\
87 & 0.7393924716163941 \\
88 & 0.7399369200295377 \\
89 & 0.7404595338546547 \\
90 & 0.7410240433442297 \\
91 & 0.7415975676521259 \\
92 & 0.7421382308824609 \\
93 & 0.7426830552371498 \\
94 & 0.7431879933445441 \\
95 & 0.7436731014086198 \\
96 & 0.7441520934020505 \\
97 & 0.7446557019131931 \\
98 & 0.7450823054354648 \\
99 & 0.7455276863801695 \\
100 & 0.7459554435434651 \\
101 & 0.7463890071512904 \\
102 & 0.7468108311404675 \\
103 & 0.7472427964906816 \\
104 & 0.7476507790380866 \\
105 & 0.7480382005495614 \\
106 & 0.7484596451446917 \\
107 & 0.748828681243284 \\
108 & 0.7492318736120062 \\
109 & 0.7496252441634682 \\
110 & 0.7499745293654959 \\
111 & 0.7502774935845222 \\
112 & 0.7506188483848721 \\
113 & 0.750952623809507 \\
114 & 0.7513482355838876 \\
115 & 0.7516710595465206 \\
116 & 0.752034080852074 \\
117 & 0.752387154766954 \\
118 & 0.7527572058612806 \\
119 & 0.753097034846578 \\
120 & 0.7534491012503808 \\
121 & 0.7537758295884472 \\
122 & 0.754112859685363 \\
123 & 0.7544539073977381 \\
124 & 0.7548071345718116 \\
125 & 0.7551368059458616 \\
126 & 0.7554713239746483 \\
127 & 0.7557666884767146 \\
128 & 0.7560692384689477 \\
129 & 0.7563713548917207 \\
130 & 0.756684312503136 \\
131 & 0.7569106943132964 \\
132 & 0.7571626484550255 \\
133 & 0.7573939118123401 \\
134 & 0.7576791882161859 \\
135 & 0.7579386005108024 \\
136 & 0.7581951144495691 \\
137 & 0.7584465422925433 \\
138 & 0.7587287473544231 \\
139 & 0.7589842090622223 \\
140 & 0.759217429305209 \\
141 & 0.7594718761856619 \\
142 & 0.7597213324856168 \\
143 & 0.7599731657369753 \\
144 & 0.7601911587425857 \\
145 & 0.7604321188278657 \\
146 & 0.7606650394364352 \\
147 & 0.7609277743829131 \\
148 & 0.761155694131533 \\
149 & 0.7613603746903499 \\
150 & 0.7615767819376869 \\
151 & 0.761767305960175 \\
152 & 0.761974705670766 \\
153 & 0.7622004505262103 \\
154 & 0.7624336471325749 \\
155 & 0.7626409809256426 \\
156 & 0.7628642253330032 \\
157 & 0.7630902997866331 \\
158 & 0.7632824376229393 \\
159 & 0.763439604941836 \\
160 & 0.7636262570134541 \\
161 & 0.7638344394032204 \\
162 & 0.7640573462020219 \\
163 & 0.7642605106184699 \\
164 & 0.7644068646680717 \\
165 & 0.7645802969637544 \\
166 & 0.764777574248241 \\
167 & 0.7649506873793912 \\
168 & 0.7651217260638186 \\
169 & 0.7652897320005917 \\
170 & 0.7654855956424038 \\
171 & 0.7656492954711596 \\
172 & 0.7657901597441059 \\
173 & 0.7659402842270909 \\
174 & 0.7661137899719879 \\
175 & 0.7662538866764615 \\
176 & 0.7664204974837465 \\
177 & 0.7665997041936939 \\
178 & 0.7667569488663231 \\
179 & 0.766890708446884 \\
180 & 0.7670463437345946 \\
181 & 0.7672053063109661 \\
182 & 0.7673624638160635 \\
183 & 0.7674986312950393 \\
184 & 0.7676429079142434 \\
185 & 0.7678311547050993 \\
186 & 0.7679896284868436 \\
187 & 0.7681224766401029 \\
188 & 0.7682904902244364 \\
189 & 0.7684295664848578 \\
190 & 0.7685653577264845 \\
191 & 0.7687007829273419 \\
192 & 0.768811903971875 \\
193 & 0.7689448298199005 \\
194 & 0.7690865865613292 \\
195 & 0.7692148438262787 \\
196 & 0.7693054267375247 \\
197 & 0.7694322852825157 \\
198 & 0.769530641800389 \\
199 & 0.7696790720895839 \\
200 & 0.7698216786383756 \\
201 & 0.769935238054598 \\
202 & 0.7700540471877241 \\
203 & 0.7701369615964961 \\
204 & 0.7702651837087893 \\
205 & 0.7703835502778045 \\
206 & 0.7704931084761545 \\
207 & 0.7706043291522233 \\
208 & 0.7707190983065859 \\
209 & 0.7708245828513945 \\
210 & 0.7709167927458394 \\
211 & 0.7710335568161013 \\
212 & 0.7711390446916068 \\
213 & 0.7712334789378263 \\
214 & 0.7713145020839065 \\
215 & 0.7714290684036621 \\
216 & 0.7714926471136339 \\
217 & 0.7715861376357368 \\
218 & 0.7716904989735301 \\
219 & 0.771779011315381 \\
220 & 0.7718685821496183 \\
221 & 0.7719719627516204 \\
222 & 0.7720711372256377 \\
223 & 0.7721751047529648 \\
224 & 0.7722722596525108 \\
225 & 0.772365984177362 \\
226 & 0.7724635854799687 \\
227 & 0.7725691872868614 \\
228 & 0.7726372755354499 \\
229 & 0.7727444454595076 \\
230 & 0.7728734623234135 \\
231 & 0.7729779392405777 \\
232 & 0.7730686312145892 \\
233 & 0.7731603926064723 \\
234 & 0.7732394073273255 \\
235 & 0.773338071583302 \\
236 & 0.7734267001885322 \\
237 & 0.7735041455062803 \\
238 & 0.773599439471459 \\
239 & 0.7737021024370713 \\
240 & 0.7738303788582134 \\
241 & 0.7739198259204397 \\
242 & 0.7739616025318359 \\
243 & 0.7740462828010317 \\
244 & 0.774137164059868 \\
245 & 0.7742325009835106 \\
246 & 0.7743033887356042 \\
247 & 0.7744036470796377 \\
248 & 0.7744865641677757 \\
249 & 0.7745740670916242 \\
250 & 0.7746953110949113 \\
251 & 0.7748019479478356 \\
252 & 0.7748473309761174 \\
253 & 0.7749107582422863 \\
254 & 0.7749941798847326 \\
255 & 0.7750647851909915 \\
256 & 0.7751555292733512 \\
257 & 0.7752377347513056 \\
258 & 0.7752985518409207 \\
259 & 0.7753875482207626 \\
260 & 0.7754486918676373 \\
261 & 0.7755476835614421 \\
262 & 0.7756447867490286 \\
263 & 0.77573707307851 \\
264 & 0.7757967483620362 \\
265 & 0.7758848702663917 \\
266 & 0.7759685649321939 \\
267 & 0.7760273819953417 \\
268 & 0.77608910069873 \\
269 & 0.7761330974937879 \\
270 & 0.7762082538041523 \\
271 & 0.7762814493700214 \\
272 & 0.7763494480465593 \\
273 & 0.7764439392341087 \\
274 & 0.7765480582891602 \\
275 & 0.7765790927756608 \\
276 & 0.7766659198182444 \\
277 & 0.7767136856726213 \\
278 & 0.7767853511168596 \\
279 & 0.776849273429316 \\
280 & 0.7769067321263116 \\
281 & 0.7769595781229313 \\
282 & 0.7770253069480494 \\
283 & 0.7770907905625627 \\
284 & 0.7771927133434466 \\
285 & 0.7772661383836148 \\
286 & 0.777330828468445 \\
287 & 0.7773951317228032 \\
288 & 0.7774632585057949 \\
289 & 0.7775531417977563 \\
290 & 0.7776247133630724 \\
291 & 0.7776892340993619 \\
292 & 0.7777569851663013 \\
293 & 0.7778229465113905 \\
294 & 0.7778663609062213 \\
295 & 0.7779283976143933 \\
296 & 0.7780159023962734 \\
297 & 0.7780833297880028 \\
298 & 0.7781763966208198 \\
299 & 0.7782314331654161 \\
300 & 0.7783153529122064 \\
301 & 0.7783746048920691 \\
302 & 0.7784614492559834 \\
303 & 0.778526982909516 \\
304 & 0.7785654055822369 \\
305 & 0.7786141276926144 \\
306 & 0.7786729952384626 \\
307 & 0.7787528286565031 \\
308 & 0.7788141677622524 \\
309 & 0.7789023788681381 \\
310 & 0.7789518497256686 \\
311 & 0.7790005169142742 \\
312 & 0.779068076602798 \\
313 & 0.7790999481384407 \\
314 & 0.7791494960633659 \\
315 & 0.7792089848803735 \\
316 & 0.7792627054293548 \\
317 & 0.7793317199657236 \\
318 & 0.7794010302258811 \\
319 & 0.7794756143588233 \\
320 & 0.7795288149062577 \\
321 & 0.7795885621859118 \\
322 & 0.7796432464124339 \\
323 & 0.7797141873460733 \\
324 & 0.7797617183355574 \\
325 & 0.7798146986680474 \\
326 & 0.7798895637724498 \\
327 & 0.7799251402080135 \\
328 & 0.7799535921152478 \\
329 & 0.7800244017937691 \\
330 & 0.7800920938739343 \\
331 & 0.7801445660250927 \\
332 & 0.7802183824221738 \\
333 & 0.7802657475786825 \\
334 & 0.7803186537656779 \\
335 & 0.7803695112031233 \\
336 & 0.7804103401792469 \\
337 & 0.7804591399629912 \\
338 & 0.7804931772085916 \\
339 & 0.7805652740230511 \\
340 & 0.780600557096007 \\
341 & 0.7806809423420793 \\
342 & 0.780726593082631 \\
343 & 0.7807826113645815 \\
344 & 0.780838873190056 \\
345 & 0.780874371664471 \\
346 & 0.780902014221243 \\
347 & 0.7809269495356911 \\
348 & 0.78098050903606 \\
349 & 0.7810302886125728 \\
350 & 0.7810843886058534 \\
351 & 0.7811203281659503 \\
352 & 0.7811785853926856 \\
353 & 0.7812103615564392 \\
354 & 0.7812580536042683 \\
355 & 0.7812917365348447 \\
356 & 0.7813399018917115 \\
357 & 0.7813950593390104 \\
358 & 0.7814581096305362 \\
359 & 0.7815058477374681 \\
360 & 0.7816092753082409 \\
361 & 0.7816512043809809 \\
362 & 0.7816804264564643 \\
363 & 0.7817276817536227 \\
364 & 0.781765559299786 \\
365 & 0.7818208051839117 \\
366 & 0.7818522790697615 \\
367 & 0.7819044139396365 \\
368 & 0.7819581769403516 \\
369 & 0.7820054736650278 \\
370 & 0.7820467340445906 \\
371 & 0.7820981357546847 \\
372 & 0.7821380239689443 \\
373 & 0.782172225781727 \\
374 & 0.7822279379229798 \\
375 & 0.782258778760369 \\
376 & 0.7823063026250661 \\
377 & 0.7823521633347379 \\
378 & 0.7824096038076677 \\
379 & 0.7824507612154473 \\
380 & 0.7824727739517452 \\
381 & 0.782512572255235 \\
382 & 0.7825460281904784 \\
383 & 0.7826160890779877 \\
384 & 0.782656158221069 \\
385 & 0.7826952219905708 \\
386 & 0.7827303322333435 \\
387 & 0.7827688273624673 \\
388 & 0.7828492785415428 \\
389 & 0.7828997582967058 \\
390 & 0.7829281662237721 \\
391 & 0.7829788829729346 \\
392 & 0.783035066687932 \\
393 & 0.783073762154376 \\
394 & 0.7831123154438493 \\
395 & 0.7831680466351743 \\
396 & 0.7831966916571289 \\
397 & 0.7832249002949219 \\
398 & 0.7832636212720355 \\
399 & 0.7832972394221811 \\
400 & 0.7833312883632739 \\
401 & 0.7833897431970479 \\
402 & 0.7834759226677098 \\
403 & 0.7835286020101804 \\
404 & 0.7835829791562647 \\
405 & 0.783625622098385 \\
406 & 0.7836711208821161 \\
407 & 0.7836979639327203 \\
408 & 0.7837363606132125 \\
409 & 0.7837966778116007 \\
410 & 0.7838181107028286 \\
411 & 0.7838511324847915 \\
412 & 0.7838832650427503 \\
413 & 0.7839379751843134 \\
414 & 0.7839812598900852 \\
415 & 0.7840502942726505 \\
416 & 0.7840838297616122 \\
417 & 0.7841549576703276 \\
418 & 0.7842205303375201 \\
419 & 0.7842569968152454 \\
420 & 0.7843293901690561 \\
421 & 0.784379272552584 \\
422 & 0.7844121164803513 \\
423 & 0.7844585410145609 \\
424 & 0.7844911990547946 \\
425 & 0.7845284904034687 \\
426 & 0.7845963908775817 \\
427 & 0.784658064239427 \\
428 & 0.7847056866876891 \\
429 & 0.7847644183898396 \\
430 & 0.7848084530888226 \\
431 & 0.7848558645222703 \\
432 & 0.7848813753274777 \\
433 & 0.784902441035595 \\
434 & 0.7849612497984031 \\
435 & 0.7850145760689504 \\
436 & 0.785062058536642 \\
437 & 0.7851187120774736 \\
438 & 0.785149381497048 \\
439 & 0.7851738814690583 \\
440 & 0.785228800956093 \\
441 & 0.7852702494797145 \\
442 & 0.7853156195414741 \\
443 & 0.785368586783918 \\
444 & 0.7853953104474507 \\
445 & 0.7854480923406173 \\
446 & 0.7854836466895235 \\
447 & 0.7855185121481081 \\
448 & 0.7855542137193959 \\
449 & 0.7855903876175448 \\
450 & 0.7856211485740574 \\
451 & 0.7856588952135053 \\
452 & 0.7857114176980795 \\
453 & 0.7857429897235733 \\
454 & 0.7857789915248958 \\
455 & 0.7858146762059011 \\
456 & 0.7858667749320093 \\
457 & 0.7858907698147736 \\
458 & 0.7859098737242977 \\
459 & 0.7859504038105175 \\
460 & 0.7859930895780345 \\
461 & 0.7860444632625082 \\
462 & 0.7860905916673675 \\
463 & 0.7861363156174372 \\
464 & 0.7861703458282322 \\
465 & 0.7862090032211903 \\
466 & 0.7862230077783207 \\
467 & 0.7862536266995112 \\
468 & 0.7862618468340025 \\
469 & 0.7863088731666323 \\
470 & 0.7863436661894989 \\
471 & 0.7863793506014027 \\
472 & 0.7864041239778987 \\
473 & 0.7864389527853395 \\
474 & 0.7864681784421921 \\
475 & 0.7865171903362831 \\
476 & 0.7865369645994965 \\
477 & 0.7865651332578546 \\
478 & 0.7865889074181226 \\
479 & 0.7866416455080564 \\
480 & 0.7866775361255366 \\
481 & 0.7867119265426756 \\
482 & 0.7867359916531877 \\
483 & 0.7867754129417609 \\
484 & 0.7868196343442547 \\
485 & 0.7868816046675564 \\
486 & 0.7869281359958522 \\
487 & 0.7869468135280838 \\
488 & 0.7869739265843322 \\
489 & 0.7869976206835712 \\
490 & 0.7870252555551567 \\
491 & 0.7870599003082943 \\
492 & 0.7870885976897932 \\
493 & 0.7871076693951962 \\
494 & 0.787131029353916 \\
495 & 0.7871717353772953 \\
496 & 0.7871883435766569 \\
497 & 0.7872645611939799 \\
498 & 0.7873139626657213 \\
499 & 0.7873438978092275 \\
500 & 0.787389377733813 \\
501 & 0.7874217880941665 \\
502 & 0.7874610740560225 \\
503 & 0.7874957902818803 \\
504 & 0.7875538064674773 \\
505 & 0.7875827694529826 \\
506 & 0.7876110309059609 \\
507 & 0.7876342836948909 \\
508 & 0.787689028999911 \\
509 & 0.7877171184527126 \\
510 & 0.7877597001170261 \\
511 & 0.7877778544011654 \\
512 & 0.7878148803458634 \\
513 & 0.7878625123972186 \\
514 & 0.7878953496433109 \\
515 & 0.7879407704011546 \\
516 & 0.7879634591530652 \\
517 & 0.7879808288289932 \\
518 & 0.7879942499994711 \\
519 & 0.7880365921534007 \\
520 & 0.788059545817586 \\
521 & 0.7881089569156572 \\
522 & 0.78815349885494 \\
523 & 0.7882230494936752 \\
524 & 0.788240803994042 \\
525 & 0.7882647459160314 \\
526 & 0.7882981705464244 \\
527 & 0.7883580283129825 \\
528 & 0.7883861109418429 \\
529 & 0.7884256055522445 \\
530 & 0.7884594560368227 \\
531 & 0.7884873328841682 \\
532 & 0.78852037887894 \\
533 & 0.7885574109951802 \\
534 & 0.7885895708551628 \\
535 & 0.7886349529206884 \\
536 & 0.7886833290735717 \\
537 & 0.7887268124815984 \\
538 & 0.7887667741258503 \\
539 & 0.788820692330868 \\
540 & 0.788842343733933 \\
541 & 0.7888799937558219 \\
542 & 0.7889071051397932 \\
543 & 0.7889292953836197 \\
544 & 0.7889733002708862 \\
545 & 0.7890179267996746 \\
546 & 0.7890561749894132 \\
547 & 0.7890776939944958 \\
548 & 0.7891096884423189 \\
549 & 0.7891575519653449 \\
550 & 0.7891989683708966 \\
551 & 0.789231667941365 \\
552 & 0.7892431103116873 \\
553 & 0.7892660995911386 \\
554 & 0.7893210370766969 \\
555 & 0.7893377291481846 \\
556 & 0.7893751035720759 \\
557 & 0.7894156240931799 \\
558 & 0.7894850030774693 \\
559 & 0.7895221980485726 \\
560 & 0.7895505032175584 \\
561 & 0.7895885853216584 \\
562 & 0.7896216295218954 \\
563 & 0.7896534913099792 \\
564 & 0.789690762671407 \\
565 & 0.7897172897483639 \\
566 & 0.789723977683374 \\
567 & 0.7897681761651254 \\
568 & 0.7898073924113943 \\
569 & 0.7898284552607785 \\
570 & 0.789855099727834 \\
571 & 0.7898935249942252 \\
572 & 0.7899090305081887 \\
573 & 0.7899364375377712 \\
574 & 0.7899727037235501 \\
575 & 0.7899986532947155 \\
576 & 0.7900418380948226 \\
577 & 0.7900738312292526 \\
578 & 0.7900804162491554 \\
579 & 0.7901274788183852 \\
580 & 0.7901656840984331 \\
581 & 0.7901921795851866 \\
582 & 0.790224750023704 \\
583 & 0.7902407089356681 \\
584 & 0.790282978396092 \\
585 & 0.7902949783222031 \\
586 & 0.7903322895068587 \\
587 & 0.7903746179291894 \\
588 & 0.7904175261207818 \\
589 & 0.7904335432380596 \\
590 & 0.7904771049334919 \\
591 & 0.7905200193966244 \\
592 & 0.7905532988543086 \\
593 & 0.7905858813305605 \\
594 & 0.790627242905388 \\
595 & 0.7906581972625323 \\
596 & 0.7906907862143138 \\
597 & 0.7907201717910609 \\
598 & 0.7907402006410463 \\
599 & 0.7907677471010723 \\
600 & 0.7907795864785775 \\
601 & 0.7908180353480582 \\
602 & 0.7908590972814553 \\
603 & 0.7908797880720496 \\
604 & 0.7909264412385628 \\
605 & 0.790967486111785 \\
606 & 0.7909997471966008 \\
607 & 0.7910320112806143 \\
608 & 0.7910524344199504 \\
609 & 0.7910847306556785 \\
610 & 0.7911225665367646 \\
611 & 0.7911500304214973 \\
612 & 0.7911712680263648 \\
613 & 0.7912095879676044 \\
614 & 0.7912615035687176 \\
615 & 0.7912819482158563 \\
616 & 0.7912872984956408 \\
617 & 0.7913153340629336 \\
618 & 0.7913466738140795 \\
619 & 0.7913677479741246 \\
620 & 0.791400508095212 \\
621 & 0.7914366740404704 \\
622 & 0.7914686294686215 \\
623 & 0.7914946221603868 \\
624 & 0.7915309145147025 \\
625 & 0.791547259209354 \\
626 & 0.7915573654798875 \\
627 & 0.791593353578909 \\
628 & 0.791608873344324 \\
629 & 0.7916243684501396 \\
630 & 0.7916570021106335 \\
631 & 0.7916970053323731 \\
632 & 0.7917226915185814 \\
633 & 0.7917381512791413 \\
634 & 0.7917591851651403 \\
635 & 0.7917753614288903 \\
636 & 0.7918064353975137 \\
637 & 0.791837645962326 \\
638 & 0.7918537871647374 \\
639 & 0.7918746823968953 \\
640 & 0.7919458386624363 \\
641 & 0.7919950479973167 \\
642 & 0.7920357552193683 \\
643 & 0.7920526501244867 \\
644 & 0.7920898510321007 \\
645 & 0.7921297260691346 \\
646 & 0.7921660338873582 \\
647 & 0.7921918365771676 \\
648 & 0.7922134836571888 \\
649 & 0.7922241314327918 \\
650 & 0.792272778114618 \\
651 & 0.7922991431473484 \\
652 & 0.7923206171727112 \\
653 & 0.7923555013396099 \\
654 & 0.7923856966907785 \\
655 & 0.7924016858361954 \\
656 & 0.7924318276972263 \\
657 & 0.7924524972360756 \\
658 & 0.7925036412954533 \\
659 & 0.7925244462419769 \\
660 & 0.7925458369925483 \\
661 & 0.7925618024222085 \\
662 & 0.7925918704013115 \\
663 & 0.7926139799643812 \\
664 & 0.7926401030427674 \\
665 & 0.7926823809570346 \\
666 & 0.7927233266203783 \\
667 & 0.7927561040402078 \\
668 & 0.7927659805396344 \\
669 & 0.7927926980332946 \\
670 & 0.7928132461585797 \\
671 & 0.7928352141936896 \\
672 & 0.7928510482427445 \\
673 & 0.7928661947418695 \\
674 & 0.7929015264594791 \\
675 & 0.7929267399137934 \\
676 & 0.7929742290837317 \\
677 & 0.7929941777704875 \\
678 & 0.7929999031469158 \\
679 & 0.7930259712136675 \\
680 & 0.7930869005302128 \\
681 & 0.7931195335841237 \\
682 & 0.7931689915971939 \\
683 & 0.7931995648907797 \\
684 & 0.793245492482262 \\
685 & 0.7932714805152329 \\
686 & 0.7932860508206903 \\
687 & 0.7933112971853499 \\
688 & 0.7933364250973607 \\
689 & 0.7933675737746316 \\
690 & 0.7933886665717634 \\
691 & 0.793445432377388 \\
692 & 0.7934805059852631 \\
693 & 0.7935097024617727 \\
694 & 0.793544947689827 \\
695 & 0.7935749963635264 \\
696 & 0.7935967431413525 \\
697 & 0.793627085434353 \\
698 & 0.7936562294196379 \\
699 & 0.7936725223125313 \\
700 & 0.7936888230872468 \\
701 & 0.7937240216090268 \\
702 & 0.7937396704486902 \\
703 & 0.7937499600733542 \\
704 & 0.7937916399370496 \\
705 & 0.7938119337204724 \\
706 & 0.7938375715471687 \\
707 & 0.7938571176608534 \\
708 & 0.7938875553889805 \\
709 & 0.7939124794339699 \\
710 & 0.7939427480335727 \\
711 & 0.7939684650154842 \\
712 & 0.7940035317219326 \\
713 & 0.7940323719945455 \\
714 & 0.7940532702677205 \\
715 & 0.7940674515065195 \\
716 & 0.7940983219300949 \\
717 & 0.7941284536542644 \\
718 & 0.7941694400160173 \\
719 & 0.7942062830665447 \\
720 & 0.7942312617207355 \\
721 & 0.7942569150672542 \\
722 & 0.7942777318746286 \\
723 & 0.7943324375280161 \\
724 & 0.7943839361318835 \\
725 & 0.7944093605450838 \\
726 & 0.7944444125320087 \\
727 & 0.7944751749900986 \\
728 & 0.7944900261058936 \\
729 & 0.7945140849493981 \\
730 & 0.7945449644871754 \\
731 & 0.7945862891166081 \\
732 & 0.7946251067349654 \\
733 & 0.794653749523664 \\
734 & 0.7946936696193537 \\
735 & 0.7947130318720383 \\
736 & 0.7947383303185447 \\
737 & 0.7947742663302017 \\
738 & 0.7948140841957729 \\
739 & 0.7948591834937454 \\
740 & 0.7948732401405686 \\
741 & 0.7949040473628787 \\
742 & 0.79492404808134 \\
743 & 0.794950025766328 \\
744 & 0.7949798984093536 \\
745 & 0.7949906287614464 \\
746 & 0.795001371689488 \\
747 & 0.7950364972969851 \\
748 & 0.7950775704715216 \\
749 & 0.7950922436801251 \\
750 & 0.7951187593995522 \\
751 & 0.7951439502443027 \\
752 & 0.7951744286813741 \\
753 & 0.7951851150980773 \\
754 & 0.7951951512495549 \\
755 & 0.7952304127726437 \\
756 & 0.7952512156100086 \\
757 & 0.7952717940328051 \\
758 & 0.7953028764710132 \\
759 & 0.795318179695096 \\
760 & 0.7953327990051468 \\
761 & 0.7953520269893221 \\
762 & 0.7953758359133764 \\
763 & 0.7954064424289563 \\
764 & 0.7954466636112231 \\
765 & 0.7954566482829896 \\
766 & 0.7954915961275073 \\
767 & 0.7955107705127716 \\
768 & 0.7955297208968647 \\
769 & 0.7955751319203389 \\
770 & 0.7955949696832132 \\
771 & 0.795634492388939 \\
772 & 0.795639875373656 \\
773 & 0.7956655584486517 \\
774 & 0.7957052592443051 \\
775 & 0.79573489870029 \\
776 & 0.7957691022153973 \\
777 & 0.7957796786578364 \\
778 & 0.7957902798868993 \\
779 & 0.7958244613605944 \\
780 & 0.7958582176746508 \\
781 & 0.7958989152678131 \\
782 & 0.7959182601228859 \\
783 & 0.7959576370104826 \\
784 & 0.7959930353319399 \\
785 & 0.7960171573602663 \\
786 & 0.7960571378886435 \\
787 & 0.7960676862233448 \\
788 & 0.7960978290530487 \\
789 & 0.7961129644025435 \\
790 & 0.7961368518926614 \\
791 & 0.7961761155510002 \\
792 & 0.7961860198718534 \\
793 & 0.7962246190508727 \\
794 & 0.7962295379349529 \\
795 & 0.7962442462826548 \\
796 & 0.7962841182149155 \\
797 & 0.7963186012065466 \\
798 & 0.796364257516068 \\
799 & 0.7963845185682092 \\
800 & 0.7964255668864386 \\
801 & 0.7964400161544637 \\
802 & 0.7964635667148192 \\
803 & 0.7964935777974155 \\
804 & 0.796512825116361 \\
805 & 0.7965272314738298 \\
806 & 0.7965672270065841 \\
807 & 0.7965916254166202 \\
808 & 0.7966308511965486 \\
809 & 0.7966616920009862 \\
810 & 0.796676089172713 \\
811 & 0.7966956667240763 \\
812 & 0.7967106951864735 \\
813 & 0.7967445334685551 \\
814 & 0.796773193249667 \\
815 & 0.7967933872012881 \\
816 & 0.7968229168297946 \\
817 & 0.7968379087313238 \\
818 & 0.7968835672915029 \\
819 & 0.7969161099539022 \\
820 & 0.7969391969012276 \\
821 & 0.7969726010633746 \\
822 & 0.79698757700746 \\
823 & 0.7970025315597862 \\
824 & 0.7970375385902109 \\
825 & 0.7970769875250814 \\
826 & 0.7970971040268958 \\
827 & 0.7971023495705449 \\
828 & 0.797116933564467 \\
829 & 0.7971463656422206 \\
830 & 0.7971712774786067 \\
831 & 0.7971765158987385 \\
832 & 0.7971868972597301 \\
833 & 0.7972072443165489 \\
834 & 0.7972370183669074 \\
835 & 0.797251938714235 \\
836 & 0.7972855431689133 \\
837 & 0.7973056080549745 \\
838 & 0.7973211195464137 \\
839 & 0.7973408354279494 \\
840 & 0.7973711758574337 \\
841 & 0.7973809210958441 \\
842 & 0.7974003411290775 \\
843 & 0.7974158266063666 \\
844 & 0.7974412562962547 \\
845 & 0.7974509969059119 \\
846 & 0.7975192842149318 \\
847 & 0.7975540903735127 \\
848 & 0.7975737568572635 \\
849 & 0.7975880185870086 \\
850 & 0.7976176538329283 \\
851 & 0.7976318936695784 \\
852 & 0.7976617962775681 \\
853 & 0.7976902055898519 \\
854 & 0.7977140511398784 \\
855 & 0.797733686324488 \\
856 & 0.7977427934921736 \\
857 & 0.7977672607540158 \\
858 & 0.7977808862338222 \\
859 & 0.7978008255758794 \\
860 & 0.7978162536509453 \\
861 & 0.7978361791944181 \\
862 & 0.7978509981499456 \\
863 & 0.797900687632407 \\
864 & 0.7979542751714196 \\
865 & 0.7980144089781109 \\
866 & 0.7980234964290288 \\
867 & 0.798038280645201 \\
868 & 0.7980626668770554 \\
869 & 0.7980870560567904 \\
870 & 0.7981018231449476 \\
871 & 0.7981456884869864 \\
872 & 0.7981838981270228 \\
873 & 0.798217941524408 \\
874 & 0.7982525239245758 \\
875 & 0.7982672858503267 \\
876 & 0.7982822856296893 \\
877 & 0.798324481242652 \\
878 & 0.7983635672292761 \\
879 & 0.7983777676817905 \\
880 & 0.7984074520574889 \\
881 & 0.7984221968780606 \\
882 & 0.7984665457647669 \\
883 & 0.7985071983613699 \\
884 & 0.7985351938125469 \\
885 & 0.7985589612769224 \\
886 & 0.7985878010317633 \\
887 & 0.7986081191456703 \\
888 & 0.7986377284009711 \\
889 & 0.7986574995132042 \\
890 & 0.7986778036411728 \\
891 & 0.7987065946117979 \\
892 & 0.7987443757638323 \\
893 & 0.7987494743044349 \\
894 & 0.7987633258686473 \\
895 & 0.7987734936140394 \\
896 & 0.798803344859189 \\
897 & 0.7988126659043421 \\
898 & 0.7988464903180582 \\
899 & 0.7988707214965518 \\
900 & 0.7988944094261069 \\
901 & 0.7989369373541632 \\
902 & 0.7989669842505026 \\
903 & 0.7989808110955365 \\
904 & 0.7990140324694708 \\
905 & 0.7990342473271816 \\
906 & 0.7990578964844572 \\
907 & 0.7990680210534722 \\
908 & 0.7990876937043127 \\
909 & 0.799116375459291 \\
910 & 0.7991445096364512 \\
911 & 0.7991782091002556 \\
912 & 0.7991973228833968 \\
913 & 0.7992354928081683 \\
914 & 0.7992492837215647 \\
915 & 0.7992824152172968 \\
916 & 0.7993020495913034 \\
917 & 0.7993603002902446 \\
918 & 0.7993836097723853 \\
919 & 0.799402152814089 \\
920 & 0.7994212214675301 \\
921 & 0.7994402860422969 \\
922 & 0.7994691328987532 \\
923 & 0.7995130313290345 \\
924 & 0.7995471441550842 \\
925 & 0.7995664600946947 \\
926 & 0.799606079283184 \\
927 & 0.7996354099005085 \\
928 & 0.7996454612866772 \\
929 & 0.7996647562629188 \\
930 & 0.7996697994011251 \\
931 & 0.7996888180309989 \\
932 & 0.7997078356130425 \\
933 & 0.7997316051695292 \\
934 & 0.7997556190106585 \\
935 & 0.7997696188808251 \\
936 & 0.7998081072634525 \\
937 & 0.799831335422898 \\
938 & 0.7998640366021357 \\
939 & 0.7998780207195242 \\
940 & 0.7998830524680838 \\
941 & 0.7998975522213506 \\
942 & 0.799921522488636 \\
943 & 0.7999410157538239 \\
944 & 0.7999660104862149 \\
945 & 0.7999859985433987 \\
946 & 0.8000239315821639 \\
947 & 0.8000518307583157 \\
948 & 0.8000755339055904 \\
949 & 0.8001094312576403 \\
950 & 0.8001423465024006 \\
951 & 0.8001523337001636 \\
952 & 0.8001765289945073 \\
953 & 0.8001994083636682 \\
954 & 0.8002188417356642 \\
955 & 0.8002516617172137 \\
956 & 0.8002862885185705 \\
957 & 0.80031521088718 \\
958 & 0.8003395884202678 \\
959 & 0.8003778451807676 \\
960 & 0.8003873030887643 \\
961 & 0.8004153973511637 \\
962 & 0.800434272118857 \\
963 & 0.8004531422923522 \\
964 & 0.800462376721546 \\
965 & 0.800485700634638 \\
966 & 0.8005045652503991 \\
967 & 0.8005328413685785 \\
968 & 0.800557156309039 \\
969 & 0.8006045313033592 \\
970 & 0.8006134607739025 \\
971 & 0.8006377560937749 \\
972 & 0.800656887891042 \\
973 & 0.8006913718176846 \\
974 & 0.8007012917826719 \\
975 & 0.8007288112376919 \\
976 & 0.800764903861785 \\
977 & 0.8007988553343802 \\
978 & 0.8008226013333033 \\
979 & 0.8008512837146857 \\
980 & 0.8008705720881174 \\
981 & 0.8008844187322778 \\
982 & 0.8009037023319917 \\
983 & 0.8009229803114546 \\
984 & 0.8009407590747222 \\
985 & 0.8009452182607247 \\
986 & 0.8009600494866186 \\
987 & 0.8009778204798832 \\
988 & 0.8010067662887026 \\
989 & 0.8010309556179537 \\
990 & 0.8010497099299897 \\
991 & 0.8010640222233973 \\
992 & 0.801101494968956 \\
993 & 0.8011295942150561 \\
994 & 0.8011586724024875 \\
995 & 0.8011729707635834 \\
996 & 0.8011867748237601 \\
997 & 0.8012102499829469 \\
998 & 0.801237822593892 \\
999 & 0.801261287668105 \\
}\expectedgaindecimili

\begin{tikzpicture}
  \begin{axis}
    \addplot table[x=values,y=P]{\expectedgaindeci};
    \addplot table[x=values,y=P]{\expectedgaincenti};
    \addplot table[x=values,y=P]{\expectedgainmili};
    \addplot table[x=values,y=P]{\expectedgaindecimili};
  \end{axis}
\end{tikzpicture}



\begin{thebibliography}{2}

\bibitem{discrete-bayes}
Robert M. Haralick,
\textit{Discrete Bayes Pattern Recognition}
\\\texttt{\url{http://haralick.org/ML/discrete_bayes.pdf}}

\bibitem{midterm-project}
Robert M. Haralick,
\textit{Midterm Project}
\\\texttt{\url{http://haralick.org/ML/midterm_project.pdf}}

\end{thebibliography}

\end{document}
